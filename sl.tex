\documentclass[12pt,letterpaper]{memoir}

\usepackage{amsmath}
\usepackage{amsfonts}
\usepackage{amssymb}
\usepackage{amsthm}
\usepackage[date=iso8601,urldate=iso8601]{biblatex}
\usepackage{centernot}
\usepackage{color}
\usepackage{datetime}
\usepackage[answerdelayed]{exercise}
% \usepackage[includehead=true,scale=0.80]{geometry}
\usepackage{graphicx}
\usepackage{listings}
\usepackage{tabu}
\usepackage{textcomp}

% \usepackage{fouriernc}
\usepackage{beramono}
\usepackage[T1]{fontenc}

\usepackage{url}
\usepackage[hidelinks]{hyperref}
\usepackage{cleveref}

\tabulinesep=1ex
\chapterstyle{thatcher}
\setlength{\parskip}{2ex}
\def\ExerciseListName{Exercise}
\yyyymmdddate
\renewcommand*{\dateseparator}{-}
\addbibresource{sl.bib}

\nocite{*}

\definecolor{mygreen}{rgb}{0.3,0.6,0.3}
\definecolor{mygray}{rgb}{0.8,0.8,0.8}
\definecolor{mymauve}{rgb}{0.58,0,0.82}
\lstset{ %
  % backgroundcolor=\color{white},   % choose the background color; you must add \usepackage{color} or \usepackage{xcolor}
  basicstyle=\footnotesize\ttfamily,     % the size of the fonts that are used for the code
  breakatwhitespace=false,         % sets if automatic breaks should only happen at whitespace
  breaklines=false,                 % sets automatic line breaking
  captionpos=b,                    % sets the caption-position to bottom
  commentstyle=\color{mygreen},    % comment style
  deletekeywords={...},            % if you want to delete keywords from the given language
  escapeinside={\%*}{*)},          % if you want to add LaTeX within your code
  extendedchars=true,              % lets you use non-ASCII characters; for 8-bits encodings only, does not work with UTF-8
  frame=single,                    % adds a frame around the code
  keepspaces=true,                 % keeps spaces in text, useful for keeping indentation of code (possibly needs columns=flexible)
  % keywordstyle=\color{blue},       % keyword style
  % Actually, we are using Idris, but Haskell is close enough
  % language=\null,                % the language of the code
  % morekeywords={*,...},            % if you want to add more keywords to the set
  numbers=left,                    % where to put the line-numbers; possible values are (none, left, right)
  numbersep=5pt,                   % how far the line-numbers are from the code
  numberstyle=\tiny\ttfamily,    % the style that is used for the line-numbers
  rulecolor=\color{mygray},        % if not set, the frame-color may be changed on line-breaks within not-black text (e.g. comments (green here))
  showspaces=false,                % show spaces everywhere adding particular underscores; it overrides 'showstringspaces'
  showstringspaces=false,          % underline spaces within strings only
  showtabs=false,                  % show tabs within strings adding particular underscores
  stepnumber=1,                    % the step between two line-numbers. If it's 1, each line will be numbered
  % stringstyle=\color{mymauve},     % string literal style
  tabsize=2,                       % sets default tabsize to 2 spaces
  title=\lstname,                   % show the filename of files included with \lstinputlisting; also try caption instead of title
  caption=\lstname ,                  % show the filename of files included with \lstinputlisting; also try caption instead of title
}

\newcommand{\monospace}[1]{{\footnotesize \path{#1}}}
\newcommand{\btrurl}[1]{{\footnotesize \url{#1}}}
\newcommand{\barelink}[1]{\monospace{<}\btrurl{#1}\monospace{>}}
\newcommand{\link}[2]{#1 \barelink{#2}}
\newcommand{\mdlink}[2]{\href{#2}{#1}}
\newcommand{\xtb}{\textbf}
\newcommand{\xti}{\emph}
% Eventually, we might have an index
\newcommand{\term}{\emph}
\newcommand{\parens}[1]{\left( #1 \right)}
\newcommand{\brackets}[1]{\left[ #1 \right]}
\newcommand{\braces}[1]{\left\{ #1 \right\}}
\newcommand{\abs}[1]{\left| #1 \right|}
\newcommand{\norm}[1]{\left\| #1 \right\|}
\newcommand{\magnitude}{\norm}
\newcommand{\N}{\mathbb{N}}
\newcommand{\R}{\mathbb{R}}
\newcommand{\Q}{\mathbb{Q}}
\newcommand{\Z}{\mathbb{Z}}
% Note that \emptyset and \varnothing are the same in newtxsf. That's
% not true in most cases
\newcommand{\nil}{\varnothing}
\newcommand{\mset}[1]{\braces{\, #1 \,}}
\newcommand{\scomp}[2]{\mset{ #1 \, : \, #2 }}

\renewcommand{\implies}{\Rightarrow}
\renewcommand{\impliedby}{\Leftarrow}
\renewcommand{\iff}{\Leftrightarrow}

\newenvironment{alignmath}[1]
  {\begin{zz}\begin{array}{#1}}
  {\end{array}\end{zz}}
\newenvironment{rcl}
  {\begin{eqnarray*}}
  {\end{eqnarray*}}
\newenvironment{rclmath}
  {\begin{rcl}}
  {\end{rcl}}
\newenvironment{zz}
  {\begin{equation*}}
  {\end{equation*}}

\newcommand{\answergraph}[1]{\includegraphics[width=0.8\textwidth]{#1}}

\theoremstyle{plain}
\newtheorem{axiom}{Axiom}[section]
\newtheorem{definition}[axiom]{Definition}
\newtheorem{theorem}[axiom]{Theorem}
\newtheorem{lemma}[axiom]{Lemma}

\theoremstyle{definition}
\newtheorem{example}[axiom]{Example}
\newtheorem{remark}[axiom]{Remark}
\newtheorem{aside}[axiom]{Aside}

\begin{document}
\title{Foundations of Pure Mathematics}
\author{Peter Harpending {\footnotesize \texttt{<peter.harpending@utah.edu>}}}
\maketitle

\newpage

The book is licensed under the Creative Commons Attribution 4.0
International License. To view a copy of this license, visit
\barelink{http://creativecommons.org/licenses/by/4.0/}.

Computer code examples found in the book are licensed under a
BSD-style license

\noindent\hrulefill

\begin{quote}
  { \footnotesize Copyright \textcopyright\ 2016, Peter Harpending.

    Redistribution and use in source and binary forms, with or without
    modification, are permitted provided that the following conditions
    are met:

    \begin{enumerate}
    \item Redistributions of source code must retain the above
      copyright notice, this list of conditions and the following
      disclaimer.

    \item Redistributions in binary form must reproduce the above
      copyright notice, this list of conditions and the following
      disclaimer in the documentation and/or other materials provided
      with the distribution.

    \item Neither the name of the copyright holder nor the names of
      its contributors may be used to endorse or promote products
      derived from this software without specific prior written
      permission.
    \end{enumerate}

    THIS SOFTWARE IS PROVIDED BY THE COPYRIGHT HOLDERS AND
    CONTRIBUTORS "AS IS" AND ANY EXPRESS OR IMPLIED WARRANTIES,
    INCLUDING, BUT NOT LIMITED TO, THE IMPLIED WARRANTIES OF
    MERCHANTABILITY AND FITNESS FOR A PARTICULAR PURPOSE ARE
    DISCLAIMED. IN NO EVENT SHALL THE COPYRIGHT HOLDER OR CONTRIBUTORS
    BE LIABLE FOR ANY DIRECT, INDIRECT, INCIDENTAL, SPECIAL,
    EXEMPLARY, OR CONSEQUENTIAL DAMAGES (INCLUDING, BUT NOT LIMITED
    TO, PROCUREMENT OF SUBSTITUTE GOODS OR SERVICES; LOSS OF USE,
    DATA, OR PROFITS; OR BUSINESS INTERRUPTION) HOWEVER CAUSED AND ON
    ANY THEORY OF LIABILITY, WHETHER IN CONTRACT, STRICT LIABILITY, OR
    TORT (INCLUDING NEGLIGENCE OR OTHERWISE) ARISING IN ANY WAY OUT OF
    THE USE OF THIS SOFTWARE, EVEN IF ADVISED OF THE POSSIBILITY OF
    SUCH DAMAGE.  }
\end{quote}

\noindent\hrulefill

\newpage
\tableofcontents

\setcounter{chapter}{-1}
\chapter{Introduction}

Hi, there! This is a textbook explaining the basic notions in pure
mathematics.  The book currently goes over basic propositional logic,
set theory, and function theory. I eventually want to expand it to go
over arithmetic in a rigorous and fun way.  If you want to jump in, go
to \cref{propositions}.

First of all, what is ``pure mathematics''. A simple definition is
``math that is not applied''. However, that is misleading, because it
might lead you to think that pure mathematics isn't applicable to the
real world. In most cases, pure math is applicable to the real
world.\footnote{That said, in many of those cases, the applications
  have yet to be discovered.} However, when you study pure math, you
usually study it because the math is interesting, not because it's
useful.

\section{Motivations and Ideology}

In American high schools, mathematics is approached as a tool for
computation, and it's almost always studied in the context of
real-world applications. This is even true of most low-level
university courses.

Here's the problem: the real world is actually rather boring. Focusing
on real-world applications usually leads people to think that math is
difficult and boring. Math can be difficult, but if you look at it the
right way, it's definitely not boring.  It's only in the second half
of an undergraduate math track that most students are immersed in pure
mathematics.

Part of this process is usually students going back and re-learning
all of the stuff they learned in high school and early college,
because they didn't learn it with enough mathematical rigor. This
means that high school and the first half of college are usually a
complete waste of math students' time.

\subsection{Target audience}

The target audience is (roughly) people who know basic and
intermediate algebra (i.e. solving static equations) and would like to
gain a deeper understanding of mathematics. This book \xti{might} be
suitable as supplementary material for people learning basic algebra.
However, it's assumed that you know how to solve reasonably simple
static equations.

In particular, I've noticed that there are a large number of
programmers who want to understand advanced topics in computer
science, but don't have a suitable mathematical background to attack
said topics effectively. Most of the early audience of this book is
people in that situation, so it's particularly suited to them.

\subsection{Contributing}

You are welcome---and encouraged---to contribute to this book. The
\LaTeX\ source for the book is on GitHub
\barelink{https://github.com/pharpend/sets-logics}, if you want to
make changes. You can also email the author at
\monospace{<peter.harpending@utah.edu>} if you find an error, or are
confused about something.

This book is free (as in ``unrestricted'') and open-source. It's also
``free'' in the sense that you don't have to pay for it, although
that's far less important.  The official license is the Creative
Commons Attribution 4.0 International License. The full license text
can be found at
\barelink{https://creativecommons.org/licenses/by/4.0/}.

You are encouraged to send your modifications back so others can use
them. You are also encouraged to publish modified versions under the
same license.

\section{Conventions}

Definitions, theorems, axioms, etc are numbered so they can be found
quickly. If I refer to ``definition X.Y.Z'', that means, the Zth
definition, in chapter X, section Y. Definitions, theorems, axioms,
etc. all use the same counter.\footnote{I took this idea from Joseph
  L. Taylor's \xti{Foundations of Analysis}.} For instance,

\begin{definition}
  Here's a definition!
\end{definition}

\begin{example}
  Here's an example.
\end{example}

\begin{axiom}
  Here's an axiom.
\end{axiom}

Exercises don't use the same counter as everything else. They are also
separated by section. For instance,

\begin{exercise}
  Here's an exercise.
\end{exercise}

\book{Basics}

\chapter{Basics}
\label{basics}

The goal of this chapter is to give you an idea of how to think
mathematically. More importantly, you should learn how to express your
ideas mathematically. The language of mathematics is extremely
powerful, and surprisingly simple.

While we are exploring mathematics, I will also display program code
examples from a variety of programming languages, to show how our
mathematical ideas translate over into programming.\footnote{The
  program code examples are licensed under the FreeBSD (2-clause)
  License, a copy of which can be found in \cref{bsd-license}.} In my
experience, learning how to program helps with math, and learning math
helps with programming.

Here are some questions that will hopefully be answered in this
chapter

\begin{enumerate}
\item What do you mean when you ``prove'' something in math? Is that
  different than ``proving'' something in real life?
\item How do I construct a mathematical proof, or even know what I'm
  supposed to prove?
\item If I'm proving something mathematically, what logical ``moves''
  am I allowed to make?
\end{enumerate}

That last question might not make a whole lot of sense right now, but
it will later on when you construct some proofs. You'll see that it's
a bit like a game of chess: you have a goal, and you have to figure
out how to get there. There are a limited number of ``moves'' you are
allowed to make to get to your goal. Sometimes, you won't be able to
figure out how to get to your goal. Sometimes, it will turn out that
it's impossible to get to your goal.

\begin{enumerate}[start=4]
\item I learned in school things like, for two numbers,
  $a + b = b + a$. How do we know that's true? Could there be some
  situation where that isn't true?
\item What is a set?
\item If we have two sets, how do we determine which one is
  ``larger''?
\item I've heard of ``cardinal numbers''. What are those? Do they have
  anything to do with the bird?
\item Can we just talk about birds instead of math?
\end{enumerate}

\Cref{s:props-proofs} and \cref{s:negation} deal with the first three
questions. \Cref{s:sets}, \cref{s:nats}, and \cref{s:other-numbers}
deal with the next two questions. \Cref{s:cardinality} deals with
questions 6 and 7. The answer to question 8 is ``no''.

\section{Propositions \& Proofs}
\label{s:props-proofs}

Thomas Goller wrote an excellent series of lecture notes for the
discrete math course taught at the University of
Utah. \cite{goller-discrete} In particular, his explanation of ``real
mathematics'' is spot-on. He references some concepts you've likely
never heard of, so I'll paraphrase it here.

Most of ``real mathematics'' is not about computing heinous sums,
doing long division of large numbers, or solving confusing word
problems, all of which you likely did in grade school. It's mostly
about taking mathematical facts, and using them to prove more
interesting mathematical facts.

There are mathematicians who do focus on modeling real-world
situations, or numerically solving difficult problems. However, the
problems they are encountering are far more difficult, and far more
interesting, than the problems you were forced to address in grade
school. Even when mathematicians do focus on real-world phenomena,
they typically invent new methods solving mathematical problems, or
analyze existing methods, rather than just solving the mathematical
problems.

On to ``mathematical facts'': what exactly is a ``mathematical fact''?
Instead of ``mathematical fact'', mathematicians usually use the term
``proposition''. A proposition can either be true or
false.\footnote{This isn't quite true. Sometimes a statement isn't
  true, and it isn't false. We'll address this issue later. For now,
  consider the phrase ``this statement is false.'' This is called the
  liar's paradox.}  They typically take two forms:

\begin{enumerate}
\item \term{Axioms}, also called \term{postulates}, are very simple
  statements that we accept without proof. You have to start
  somewhere.
\item \term{Theorems}, also called \term{lemmas}, are facts that can
  be proven from other axioms or theorems.
\end{enumerate}

Most theorems are proved from other theorems, and not directly from
the axioms. You generally want to minimize the number of axioms you
use, in the interest of making your argument more
believable.\footnote{This is a principle known as ``Occam's Razor''.}
Nothing would ever get done if we had to start from axioms every time.

\subsection{Logical implication}

With that out of the way, let's address our first topic:
\term{implication}. If we want to say that some proposition $A$
implies another proposition $B$, we write $$A \implies B,$$ which
should be read as ``$A$ implies $B$.''\footnote{The ``double arrow''
  notation $A \Rightarrow B$ is far more common. Later, you'll see why
  we use a single arrow.}  You should note that $A \implies B$ is
itself a proposition.

So, what does this mean? Well, $$A \implies B$$ is equivalent to
saying \ctext{if $A$ then $B$.} The expression $B \impliedby A$ means
the same thing as $A \implies B.$ Sometimes, you'll see $$A \iff B,$$
which means \ctext{$A$ implies $B$, \xtb{and} $B$ implies $A$.} This
should be read as ``$A$ if and only if $B$.'' Sometimes, you'll see
the word \ctext{iff,} which isn't a word in English, but it's
shorthand for \ctext{if and only if.}

Think about this: $$A \implies B$$ means \ctext{if $A$ then $B$.}
However, this could also be stated \ctext{$B$ if $A$.} Likewise, it
could also be stated \ctext{$A$ only if $B$.} Hence, $A \iff B$ is
usually said as ``$A$ if and only if $B$.''

\subsection{Logical conjunction and disjunction}

\term{Logical conjunction} is a fancy term for the word
``and''. Sometimes, you'll see the notation $$A \land B,$$ which means
the same thing as \ctext{$A$ logical-and $B$.} In this book, it's easy
enough to just use text, so I'll just write $$A \tand B.$$ There's no
real difference between ``logical-and'' and the normal way you use
``and'' in everyday life.

\emph{However}, there's a huge difference between \term{logical-or},
and the colloquial meaning of ``or,'' which is why we have the prefix
``logical.'' What is ``logical-or,'' then? Consider the phrase
\ctext{$A$ or $B$.} In everyday life, this means that you can either
have $A$, or you can have $B$, but you can't have both, and you can't
have neither. In logic, the phrase \ctext{$A$ or $B$} allows you to
have both.

\section{Negation}
\label{s:negation}

\section{Sets}
\label{s:sets}

\section{Natural numbers}
\label{s:nats}
\label{s:induction}

\section{Unnatural numbers}
\label{s:unnats}
\label{s:other-numbers}

\section{Functions}
\label{s:functions}

\section{Cardinality}
\label{s:cardinality}
\chapter{Sets}

Sets are one of the most basic mathematical tools. The purpose of this
chapter is to define sets, and do basic reasoning about them. We're
also going to give a loose definition of common sets we deal
with. Spoiler: it will later turn out that much of what we do in this
chapter is actually wrong, and can lead to some
contradictions.\footnote{If you're really impatient, and can't wait
  for another chapter, you can read the \link{Wikipedia page on
    Russell's
    paradox}{https://en.wikipedia.org/wiki/Russell\%27s_paradox}}

Sets are collections of objects. The easiest way to denote sets is to
list their elements between curly braces: $\braces{}$.

\begin{displaymath}
  \mset{1,2,3,4}
\end{displaymath}

There is no notion of duplication or order, so all of the following
sets are the same

\begin{tabu}{rp{5.5cm}}
  $\mset{1,2,3,4}$         & \\
  $\mset{1,1,2,2,3,3,4,4}$ & (Each element is duplicated.) \\
  $\mset{2,1,4,3,4,4,1}$   & (The order is different, and some of the elements are duplicated.)
\end{tabu}

There's a set with no elements, called the ``null set'', or the
``empty set'', or a variety of other names. It's usually denoted
$\emptyset$, $\phi$, or some variant thereof. I'm going to use $\nil$.

First of all, if we have a set $A$ and an object $x$, and $x$ is an
element of $A$, we'll write $x \in A$. Note: I've have an old textbook
that uses the Greek letter epsilon ($\epsilon$ or $\varepsilon$) in
place of $\in$. The $\in$ symbol kind of looks like an `E', which you
can mentally associate with ``element''. The symbol $\in$ can be read
as ``is an element of''. If some potential element $y$ is \xtb{not} in
$A$, then we'll write $y \notin A$.\footnote{As a general rule, if you
  have some symbol that indicates that something is true, crossing out
  the symbol indicates that thing is false.}

Continuing with the above example, we have the set
$\mset{1,2,3,4}$. We have

\begin{displaymath}
  1 \in \mset{1,2,3,4}
\end{displaymath}

but

\begin{displaymath}
  5 \notin \mset{1,2,3,4}
\end{displaymath}

We can have sets that have infinitely many elements. For instance, the
``natural numbers'', usually denoted with $\N$, are
infinite.\footnote{Conventions differ on whether or not $\N$ includes
  $0$. It doesn't matter all that much, especially as far as this book
  is concerned.} These are positive whole numbers (i.e. not fractions
or decimals).

\begin{displaymath}
  \N = \mset{1,2,3,4,\dots}
\end{displaymath}

The ``integers'' are any whole numbers, which include $0$ and negative
numbers. The integers are usually denoted with $\Z$. The $\Z$ stands
for ``Zahlen'', which is German for ``numbers''.\footnote{I heard this
  from a professor once. If it's wrong, blame him.}

\begin{displaymath}
  \Z = \mset{\dots, -3, -2, -1, 0, 1, 2, 3, \dots}
\end{displaymath}

\section{Set comprehensions}

Before we go much further, I want to introduce the notion of a ``set
comprehension'', or ``set builder notation''. The basic format is this:

\begin{displaymath}
  \scomp{\text{\xtb{What each element looks like}}}{\text{\xtb{Conditions}}}
\end{displaymath}

The ``conditions'' are just things that have to hold true about the
element.

Here are some examples:

\begin{displaymath}
  \scomp{x \in \N}{x < 5} = \mset{1,2,3,4}
\end{displaymath}
\begin{displaymath}
  \scomp{2x \in \N}{x < 5} = \mset{2,4,6,8}
\end{displaymath}
\begin{displaymath}
  \scomp{x \in \Z}{x < 5} = \mset{\dots,-3,-2,-1,0,1,2,3,4}
\end{displaymath}

Each set comprehension should be read in two parts: the part before
the colon, and the part after the colon.

\begin{itemize}
\item $x \in \N$ means that we are choosing\footnote{Yes, we'll get to
    the axiom of choice later. Calm down.} elements of $\N$, and we're
  calling the element $x$.
\item $x < 5$ means that we are choosing all elements of $\N$ which
  are less than $5$. In the part before the colon, we are choosing one
  such element as an example, and calling it $x$.
\end{itemize}

\subsection{Trying these on your computer}

Many programming languages have ``list comprehensions'', which are
conceptually similar to set comprehensions. For instance, in Haskell,
we can construct the equivalent of our sets from above like this:

\begin{lstlisting}[language=Haskell,caption={The sets from above, in Haskell}]
ghci> [ x | x <- [1..4] ]
[1,2,3,4]
ghci> [ 2*x | x <- [1..4] ]
[2,4,6,8]
\end{lstlisting}

In Python, the syntax is a bit different, but:

\begin{lstlisting}[language=Python,caption={The same thing in Python}]
>>> [ x for x in range(1,5) ]
[1, 2, 3, 4]
>>> [ 2*x for x in range(1,5) ]
[2, 4, 6, 8]
\end{lstlisting}

You probably already have Python installed on your system. Open a
terminal, and run the command \monospace{python} (hit Enter after
typing ``python''). If not, try following the instructions here:
\barelink{https://wiki.python.org/moin/BeginnersGuide/Download}.

You probably do \xtb{not} have Haskell installed on your system,
unless you installed it on purpose. You can find documentation on how
to install Haskell here: \barelink{http://haskellstack.org/}.

\subsubsection{Lists v. Sets}

Lists are different from sets, because, in a list, order and
duplication matter. You can test this out in Haskell like this:

\begin{lstlisting}[language=Haskell,caption={List equality in Haskell}]
ghci> [1,2,3,4] == [1,2,3,4]
True
ghci> [1,2,3,4] == [4,3,2,1]
False
ghci> [1,2,3,4] == [1,2,3,1,3,3,1,4,2]
False
\end{lstlisting}

Haskell actually has a type for sets, called \monospace{Set}. To turn
a list into a set, you use the function \monospace{fromList}

\begin{lstlisting}[language=Haskell,caption={Set equality in Haskell}]
ghci> import Data.Set
ghci> fromList [1,2,3,4] == fromList [1,2,3,4]
True
ghci> fromList [1,2,3,4] == fromList [4,3,2,1]
True
ghci> fromList [1,2,3,4] == fromList [1,4,3,1,2]
True
\end{lstlisting}

\section{Rational and Real Numbers}

The reason I introduced the set comprehension notation is because (1)
it's useful and important, and (2) it's convenient to define the
rational numbers (numbers that can be expressed as a ratio of two
integers) like this:

\begin{displaymath}
  \Q = \scomp{\frac{x}{y} \in \R}{x, y \in \Z; y \ne 0}
\end{displaymath}

Remember, kids: \xtb{you can't divide by zero}, hence the $y \ne 0$
part. The notation $x, y \in \Z$ is just shorthand for $x \in \Z$ and
$y \in \Z$. Mathematicians are incredibly lazy, so you'll often see
confusing notational shorthand used as a replacement for actually
explaining an idea. The symbol $\R$ refers to the ``real
numbers''. The symbol $\Q$ stands for ``quotient''.

Rational numbers are numbers that can be expressed as a ratio of two
integers. Real numbers turn out to be really hard to define. You can
for now think of a real number as any number that can be written down,
supposing one had infinite paper, ink, and time. For instance, $\pi$
cannot be expressed as the ratio of two integers, but is a real
number.

\subsection{Numerical representations of Real numbers}

As it turns out, this definition of real numbers doesn't work! The
reason is, there are cases where you write down two different numbers,
and they refer to the same number.

(I stole this example from a professor of mine, Peter Alfeld, who
presumably stole it from someone else.\cite{pa-unique})

\begin{example}[Issues with numerical representations of real numbers]
  Consider the number $x = 0.9999999\dots$. We can multiply by
  $10$. To do so, we just move everything one digit to the left.

  \begin{rclmath}
      x & = & 0.99999\dots \\
    10x & = & 9.99999\dots \\
  \end{rclmath}

  Well, we can take $9x = 10x - x$, and get this

  \begin{rclmath}
    10x & = & 9.99999\dots \\
      x & = & 0.99999\dots \\
     9x & = & 9 \\
  \end{rclmath}

  We get this because every single digit to the right of the decimal
  point cancels. We can take $x = \frac{9x}{9}$ and get

  \begin{rclmath}
     9x & = & 9 \\
      x & = & 1 \\
  \end{rclmath}

  This is a silly example, and doesn't constitute a ``proof''. However,
  it should give you a glimpse at why numerical representations of real
  numbers are problematic.
\end{example}

Here's another example, which I stole from LeRoy Eide (he showed me
this in person).

\begin{example}[Issues with numerical representations of integers]
  Take a number that begins with an endless string of $9$s to the
  left.

  \begin{displaymath}
    x = \dots99994
  \end{displaymath}

  Let's take $x + 6$.

  \begin{displaymath}
    \begin{tabu}{rr}
        & \dots99994 \\
      + & 6 \\
      \tabucline \\
      = & 0 \\
    \end{tabu}
  \end{displaymath}

  Try it! You add the $6$ to the $4$, and get $10$. Fair enough, put
  $0$ and carry the $1$

  \tabulinesep=0ex
  \begin{displaymath}
    \begin{tabu}{rr}
        & \text{1\hspace{2.1mm}}  \\
        & \dots99994 \\
      + & 6 \\
      \tabucline \\
      = & 00 \\
    \end{tabu}
  \end{displaymath}

  Let's try it again

  \begin{displaymath}
    \begin{tabu}{rr}
        & \text{1\hspace{0.9mm}1\hspace{2.1mm}}  \\
        & \dots99994 \\
      + & 6 \\
      \tabucline \\
      = & 000 \\
    \end{tabu}
  \end{displaymath}

  Continue this process...

  \begin{displaymath}
    \begin{tabu}{rr}
        & \text{\dots 1\hspace{0.9mm}1\hspace{0.9mm}1\hspace{0.9mm}1\hspace{2.1mm}}  \\
        & \dots99994 \\
      + & 6 \\
      \tabucline \\
      = & \dots 00000 \\
    \end{tabu}
  \end{displaymath}

  Therefore, we conclude that $\dots999994 + 6 = 0$, and therefore
  that $\dots99994 = -6$.
\end{example}

Note, we do use infinite carry all the time. For instance, it's why

\begin{displaymath}
  \frac{2}{3} + \frac{1}{3} = 0.666\dots + 0.333\dots = 0.999\dots = 1
\end{displaymath}

Infinite carry actually follows nicely from something called
``mathematical induction''. I'll explain induction in a later chapter.

\begin{example}(More insanity with numbers)
  We can take this a step further, and combine the two cases.

  Let's look at $x = \dots99994$, and examine what happens when we
  take $\frac{x}{100}$.

  \begin{displaymath}
    \frac{x}{100} = \dots999.94
  \end{displaymath}

  Let's look at $\frac{x}{10}$

  \begin{displaymath}
    \frac{x}{10} = \dots999.4
  \end{displaymath}

  Well, remember $\frac{x}{10} = \frac{10x}{100}$. Let's let
  $y = \frac{x}{100}$. Therefore, $10y = \frac{x}{10}$. Let's do what
  we did before, and take $10y - y$

  \begin{displaymath}
    \begin{tabu}{rr}
        & \dots999.4\text{\hspace{2mm}} \\
      - & \dots999.94 \\
      \tabucline \\
      = & -0.54 \\
      = & -\frac{54}{100} \\
    \end{tabu}
  \end{displaymath}

  Now we have that $9y = -\frac{54}{100}$. Therefore, $y =
  -\frac{6}{100} = -\frac{3}{50}$. However, we had that $y =
  \frac{x}{10}$, and therefore $10y = x$. Thus, we have

  \begin{displaymath}
    x = -\frac{3}{5} = -0.6
  \end{displaymath}

  Remember, before, we proved $x = -6$.
\end{example}

The problem here is that \xtb{decimal (base $10$) representations of
  real numbers are not unique}. That sounds like a contradiction, but
it really isn't. It turns out, and we might prove this later, that you
can tweak the rules of writing numbers down to make decimal
representations unique. For decimal numbers, we assert that the number
can't begin or end with an infinite string of $9$s.

\section{Unions and Intersections}
\section{Differences}
\section{De Morgan's Laws}


\chapter{Functions}
  \section{Domain}
  \section{Codomain}
  \section{Range}
  \section{Recursion}
  \section{Currying}

\chapter{Proofs}
  \section{Propositions}
  \section{Proofs}

\chapter{Rigorous set theory}
  \section{Functions}
  \section{Types}
  \section{Curry-Howard correspondence}

\chapter{Groups}
\chapter{Rings}
\chapter{Fields}
\chapter{Vector Spaces}


\printbibliography
\end{document}
