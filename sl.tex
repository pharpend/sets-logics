\documentclass[12pt,oneside,ebook]{memoir}

\usepackage{amsmath}
\usepackage{amsfonts}
\usepackage{amssymb}
\usepackage{amsthm}
\usepackage[date=iso8601,urldate=iso8601]{biblatex}
\usepackage{centernot}
\usepackage{color}
\usepackage{datetime}
\usepackage[answerdelayed]{exercise}
\usepackage{graphicx}
\usepackage{listings}
\usepackage{tabu}

\usepackage[T1]{fontenc}
\usepackage{newtxtext}
\usepackage[scaled=0.85]{FiraSans}
% \usepackage{beramono}
\usepackage{newtxsf}
% \usepackage{newtxmath}
\renewcommand*{\familydefault}{\sfdefault}

\usepackage{url}
\usepackage[hidelinks]{hyperref}
\usepackage{cleveref}

\tabulinesep=1ex
\chapterstyle{bianchi}
\setlength{\parskip}{2ex}
\def\ExerciseListName{Exercise}
\yyyymmdddate
\renewcommand*{\dateseparator}{-}
\addbibresource{sl.bib}

\nocite{*}

\definecolor{mygreen}{rgb}{0.3,0.6,0.3}
\definecolor{mygray}{rgb}{0.8,0.8,0.8}
\definecolor{mymauve}{rgb}{0.58,0,0.82}
\lstset{ %
  % backgroundcolor=\color{white},   % choose the background color; you must add \usepackage{color} or \usepackage{xcolor}
  basicstyle=\footnotesize\ttfamily,     % the size of the fonts that are used for the code
  breakatwhitespace=false,         % sets if automatic breaks should only happen at whitespace
  breaklines=false,                 % sets automatic line breaking
  captionpos=b,                    % sets the caption-position to bottom
  commentstyle=\color{mygreen},    % comment style
  deletekeywords={...},            % if you want to delete keywords from the given language
  escapeinside={\%*}{*)},          % if you want to add LaTeX within your code
  extendedchars=true,              % lets you use non-ASCII characters; for 8-bits encodings only, does not work with UTF-8
  frame=single,                    % adds a frame around the code
  keepspaces=true,                 % keeps spaces in text, useful for keeping indentation of code (possibly needs columns=flexible)
  % keywordstyle=\color{blue},       % keyword style
  % Actually, we are using Idris, but Haskell is close enough
  % language=\null,                % the language of the code
  % morekeywords={*,...},            % if you want to add more keywords to the set
  numbers=left,                    % where to put the line-numbers; possible values are (none, left, right)
  numbersep=5pt,                   % how far the line-numbers are from the code
  numberstyle=\footnotesize\ttfamily,    % the style that is used for the line-numbers
  rulecolor=\color{mygray},        % if not set, the frame-color may be changed on line-breaks within not-black text (e.g. comments (green here))
  showspaces=false,                % show spaces everywhere adding particular underscores; it overrides 'showstringspaces'
  showstringspaces=false,          % underline spaces within strings only
  showtabs=false,                  % show tabs within strings adding particular underscores
  stepnumber=1,                    % the step between two line-numbers. If it's 1, each line will be numbered
  % stringstyle=\color{mymauve},     % string literal style
  tabsize=2,                       % sets default tabsize to 2 spaces
  title=\lstname,                   % show the filename of files included with \lstinputlisting; also try caption instead of title
  caption=\lstname ,                  % show the filename of files included with \lstinputlisting; also try caption instead of title
}

\newcommand{\monospace}[1]{{\footnotesize \texttt{#1}}}
\newcommand{\btrurl}[1]{{\footnotesize \url{#1}}}
\newcommand{\barelink}[1]{\monospace{<}\btrurl{#1}\monospace{>}}
\newcommand{\link}[2]{#1 \barelink{#2}}
\newcommand{\mdlink}[2]{\href{#2}{#1}}
\newcommand{\anauthor}[2]{#1 \monospace{<{#2}>}}
\newcommand{\xtb}{\textbf}

\newcommand{\parens}[1]{\left( #1 \right)}
\newcommand{\brackets}[1]{\left[ #1 \right]}
\newcommand{\braces}[1]{\left\{ #1 \right\}}
\newcommand{\abs}[1]{\left| #1 \right|}
\newcommand{\norm}[1]{\left\| #1 \right\|}
\newcommand{\magnitude}{\norm}
\newcommand{\N}{\mathbb{N}}
\newcommand{\R}{\mathbb{R}}
\newcommand{\Q}{\mathbb{Q}}
\newcommand{\Z}{\mathbb{Z}}
% Note that \emptyset and \varnothing are the same in newtxsf. That's
% not true in most cases
\newcommand{\nil}{\varnothing}
\newcommand{\mset}[1]{\braces{\, #1 \,}}
\newcommand{\scomp}[2]{\mset{ #1 \, : \, #2 }}

\newenvironment{alignmath}[1]
  {\begin{displaymath}\begin{array}{#1}}
  {\end{array}\end{displaymath}}
\newenvironment{rclmath}
  {\begin{alignmath}{rcl}}
  {\end{alignmath}}

\newcommand{\answergraph}[1]{\includegraphics[width=0.8\textwidth]{#1}}

\theoremstyle{definition}
\newtheorem{example}{Example}[section]

\begin{document}
\title{Sets and Logics}
\author{\anauthor{Peter Harpending}{peter.harpending@utah.edu}}
\maketitle

This work is licensed under the Creative Commons Attribution 4.0
International License. To view a copy of this license, visit
\barelink{http://creativecommons.org/licenses/by/4.0/}.

You can find the source for this book on GitHub
\barelink{https://github.com/pharpend/sets-logics}.

\newpage
\tableofcontents

\chapter{Sets}

Sets are one of the most basic mathematical tools. The purpose of this
chapter is to define sets, and do basic reasoning about them. Spoiler:
it will later turn out that much of what we do in this chapter is
actually wrong, and can lead to some contradictions.\footnote{If
  you're really impatient, and can't wait for another chapter, you can
  read the \link{Wikipedia page on Russell's
    paradox}{https://en.wikipedia.org/wiki/Russell\%27s_paradox}}

Sets are collections of objects. The easiest way to denote sets is to
list their elements between curly braces: $\braces{}$.

\begin{displaymath}
  \mset{1,2,3,4}
\end{displaymath}

There is no notion of duplication or order, so all of the following
sets are the same

\begin{tabu}to \linewidth {rp{5.5cm}}
  $\mset{1,2,3,4}$         & \\
  $\mset{1,1,2,2,3,3,4,4}$ & (Each element is duplicated.) \\
  $\mset{2,1,4,3,4,4,1}$   & (The order is different, and some of the elements are duplicated.)
\end{tabu}

There's a set with no elements, called the ``null set'', or the
``empty set'', or a variety of other names. It's usually denoted
$\emptyset$, $\varnothing$, $\phi$, or some variant of that
symbol. I'm going to use $\nil$.


First of all, if we have a set $A$ and an object $x$, and $x$ is an
element of $A$, we'll write $x \in A$. Note: I've have an old textbook
that uses the Greek letter epsilon ($\epsilon$) in place of $\in$. The
$\in$ symbol kind of looks like an `E', which you can mentally
associate with ``element''. The symbol $\in$ can be read as ``is an
element of''. If some potential element $y$ is \xtb{not} in $A$, then
we'll write $y \notin A$.\footnote{As a general rule, if you have some
  symbol that indicates that something is true, crossing out the
  symbol indicates that thing is false.}

Continuing with the above example, we have the set
$\mset{1,2,3,4}$. We have

\begin{displaymath}
  1 \in \mset{1,2,3,4}
\end{displaymath}

but 

\begin{displaymath}
  5 \notin \mset{1,2,3,4}
\end{displaymath}

We can have sets that have infinitely many elements. For instance, the
``natural numbers'', usually denoted with $\N$, are
infinite.\footnote{Conventions differ on whether or not $\N$ includes
  $0$. It doesn't matter all that much, especially as far as this book
  is concerned.} These are positive whole numbers (i.e. not fractions
or decimals).

\begin{displaymath}
  \N = \mset{1,2,3,4,\dots}
\end{displaymath}

The ``integers'' are any whole numbers, which include $0$ and negative
numbers. The integers are usually denoted with $\Z$. The $\Z$ stands
for ``Zahlen'', which is German for ``numbers''.\footnote{I heard this
  from a professor once. If it's wrong, blame him.}

\begin{displaymath}
  \Z = \mset{\dots, -3, -2, -1, 0, 1, 2, 3, \dots}
\end{displaymath}

Before we go much further, I want to introduce the notion of a ``set
comprehension'', or ``set builder notation''.

\begin{displaymath}
  \scomp{x \in \N}{x < 5} = \mset{1,2,3,4}
\end{displaymath}
\begin{displaymath}
  \scomp{x \in \Z}{x < 5} = \mset{\dots,-3,-2,-1,0,1,2,3,4}
\end{displaymath}

Each set comprehension should be read in two parts: the part before
the colon, and the part after the colon.

\begin{itemize}
\item $x \in \N$ means that we are choosing\footnote{Yes, we'll get to
    the axiom of choice later. Calm down.} elements of $\N$, and we're
  calling the element $x$.
\item $x < 5$ means that we are choosing all elements of $\N$ which
  are less than $5$. In the part before the colon, we are choosing one
  such element as an example, and calling it $x$.
\end{itemize}

\chapter{Functions}
  \section{Recursion}

\chapter{Proofs}
  \section{Propositions}
  \section{Proofs}

\chapter{Curry-Howard correspondence}

\begin{appendices}
  \chapter{Solutions to the exercises}
  \shipoutAnswer
\end{appendices}

\printbibliography
\end{document}
