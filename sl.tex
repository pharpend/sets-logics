\usepackage{amsmath}
\usepackage{amsfonts}
\usepackage{amssymb}
\usepackage{amsthm}
\usepackage{appendix}
\usepackage[backend=bibtex,date=iso8601,urldate=iso8601]{biblatex}
\usepackage{centernot}
\usepackage{color}
\usepackage{datetime}
\usepackage{enumitem}
\usepackage[answerdelayed]{exercise}
\usepackage{graphicx}
\usepackage{listings}
\usepackage{tabu}
\usepackage{textcomp}

\usepackage{url}
\usepackage[hidelinks]{hyperref}
\usepackage{cleveref}

\tabulinesep=1ex
\def\ExerciseListName{Exercise}
\yyyymmdddate
\renewcommand*{\dateseparator}{-}
\addbibresource{sl.bib}

\nocite{*}

\definecolor{mygreen}{rgb}{0.3,0.6,0.3}
\definecolor{mygray}{rgb}{0.8,0.8,0.8}
\definecolor{mymauve}{rgb}{0.58,0,0.82}
\lstset{ %
  % backgroundcolor=\color{white},   % choose the background color; you must add \usepackage{color} or \usepackage{xcolor}
  basicstyle=\tiny\ttfamily,     % the size of the fonts that are used for the code
  breakatwhitespace=false,         % sets if automatic breaks should only happen at whitespace
  breaklines=true,                 % sets automatic line breaking
  captionpos=b,                    % sets the caption-position to bottom
  commentstyle=\color{mygreen},    % comment style
  deletekeywords={...},            % if you want to delete keywords from the given language
  escapeinside={\%*}{*)},          % if you want to add LaTeX within your code
  extendedchars=true,              % lets you use non-ASCII characters; for 8-bits encodings only, does not work with UTF-8
  frame=single,                    % adds a frame around the code
  keepspaces=true,                 % keeps spaces in text, useful for keeping indentation of code (possibly needs columns=flexible)
  % keywordstyle=\color{blue},       % keyword style
  % Actually, we are using Idris, but Haskell is close enough
  % language=\null,                % the language of the code
  % morekeywords={*,...},            % if you want to add more keywords to the set
  numbers=left,                    % where to put the line-numbers; possible values are (none, left, right)
  numbersep=5pt,                   % how far the line-numbers are from the code
  numberstyle=\tiny\ttfamily,    % the style that is used for the line-numbers
  postbreak=\raisebox{0ex}[0ex][0ex]{\ensuremath{\color{red}\hookrightarrow\space}},
  rulecolor=\color{mygray},        % if not set, the frame-color may be changed on line-breaks within not-black text (e.g. comments (green here))
  showspaces=false,                % show spaces everywhere adding particular underscores; it overrides 'showstringspaces'
  showstringspaces=false,          % underline spaces within strings only
  showtabs=false,                  % show tabs within strings adding particular underscores
  stepnumber=1,                    % the step between two line-numbers. If it's 1, each line will be numbered
  % stringstyle=\color{mymauve},     % string literal style
  tabsize=2,                       % sets default tabsize to 2 spaces
  title=\lstname,                   % show the filename of files included with \lstinputlisting; also try caption instead of title
  caption=\lstname ,                  % show the filename of files included with \lstinputlisting; also try caption instead of title
}

\newcommand{\monospace}[1]{{\footnotesize \path{#1}}}
\newcommand{\btrurl}[1]{{\footnotesize \url{#1}}}
\newcommand{\barelink}[1]{\monospace{<}\btrurl{#1}\monospace{>}}
\newcommand{\link}[2]{#1 \barelink{#2}}
\newcommand{\mdlink}[2]{\href{#2}{#1}}
\newcommand{\xtb}{\textbf}
\newcommand{\xti}{\emph}
% Eventually, we might have an index
\newcommand{\term}{\xtb}
\newcommand{\parens}[1]{\left( #1 \right)}
\newcommand{\brackets}[1]{\left[ #1 \right]}
\newcommand{\braces}[1]{\left\{ #1 \right\}}
\newcommand{\abs}[1]{\left| #1 \right|}
\newcommand{\norm}[1]{\left\| #1 \right\|}
\newcommand{\magnitude}{\norm}
\newcommand{\N}{\mathbb{N}}
\newcommand{\R}{\mathbb{R}}
\newcommand{\Q}{\mathbb{Q}}
\newcommand{\Z}{\mathbb{Z}}
% Note that \emptyset and \varnothing are the same in newtxsf. That's
% not true in most cases
\newcommand{\nil}{\varnothing}
\newcommand{\mset}[1]{\braces{\, #1 \,}}
\newcommand{\scomp}[2]{\mset{ #1 \, : \, #2 }}
\newcommand{\ctext}[1]{$$\text{#1}$$}
\newcommand{\tand}{\text{ and }}
\newcommand{\tor}{\text{ or }}
\newcommand{\True}{\text{True}}
\newcommand{\False}{\text{False}}

\renewcommand{\implies}{\to}
\renewcommand{\impliedby}{\leftarrow}
\renewcommand{\iff}{\leftrightarrow}

\newenvironment{alignmath}[1]
  {\begin{zz}\begin{tabu}{#1}}
  {\end{tabu}\end{zz}}
\newenvironment{rcl}
  {\begin{alignmath}{rcl}}
  {\end{alignmath}}
\newenvironment{rclmath}
  {\begin{rcl}}
  {\end{rcl}}
\newenvironment{zz}
  {\begin{equation*}}
  {\end{equation*}}

\newcommand{\answergraph}[1]{\includegraphics[width=0.8\textwidth]{#1}}

\theoremstyle{plain}
\newtheorem{axiom}{Axiom}[section]
\newtheorem{definition}[axiom]{Definition}
\newtheorem{theorem}[axiom]{Theorem}
\newtheorem{lemma}[axiom]{Lemma}

\theoremstyle{definition}
\newtheorem{example}[axiom]{Example}
\newtheorem{remark}[axiom]{Remark}
\newtheorem{aside}[axiom]{Aside}
\newtheorem{exercise}{Exercise}[section]

\begin{document}
\title{Foundations of Pure Mathematics}
\author{Peter Harpending {\footnotesize \texttt{<peter.harpending@utah.edu>}}}
\maketitle

\noindent\hrulefill

The book is licensed under the Creative Commons Attribution 4.0
International License. To view a copy of this license, visit
\barelink{http://creativecommons.org/licenses/by/4.0/}.

\noindent\hrulefill

\tableofcontents

\chapter{Basics}
\label{basics}

\sillygraph{images/f-mustache-2-gimped.png} The goal of this chapter
is to give you an idea of how to think mathematically. More
importantly, you should learn how to express your ideas
mathematically. The language of mathematics is extremely powerful, and
surprisingly simple.


\subsection{Programming}

While we are exploring mathematics, I will also display program code
examples from a variety of programming languages, (but mostly Haskell)
to show how our mathematical ideas translate over into
programming.\footnote{The program code examples are licensed under the
  FreeBSD (2-clause) License, a copy of which can be found in
  \cref{bsd-license}.} In my experience, learning how to program helps
with math, and learning math helps with programming.

In particular, Haskell is a \term{functional} programming language,
meaning that you program entirely with \term{mathematical
  functions}. Hey, here's our first informal definition!

\begin{definition}
  A \term{mathematical function} is something that takes some input,
  and gives a deterministic output. That is, given a function $f$, and
  two objects $a$ and $b$, \ctext{if $a = b$, then $f(a) = f(b)$.}
\end{definition}

Don't worry if you don't understand all of this right now, it will
make a lot more sense later. Just keep moving.

\begin{example}
  Here's a function that takes a number and adds $2$ to it: $$f(x) =
  x + 2.$$
\end{example}

The defining characteristic of a \emph{mathematical} function is,
\textbf{if you call a function twice with the same arguments, you
  should get the same result}. This is called \term{referential
  transparency}.  This \emph{might} seem obvious. However, in most
programming languages, it doesn't hold up. For instance, consider this
seemingly innocuous code in Python's interactive console:

\begin{lstlisting}[language={Python}]
>>> import random
>>> random.randint(0, 5)
3
>>> random.randint(0, 5)
4
\end{lstlisting}

I called the same ``function'' twice, with the same arguments, and got
a different result. Therefore, \monospace{random.randint} is \xtb{not}
a mathematical function. Haskell works exclusively with mathematical
functions.

\question{How do you do things like tell the time?}

A common thing you need to do while programming is figure out what
time it is. In Python, this is pretty easy

\begin{lstlisting}[language=Python]
>>> import time
>>> print(time.strftime('The time right now is %-l:%M:%S %p.'))
The time right now is 4:28:01 PM.
>>> print(time.strftime('The time right now is %-l:%M:%S %p.'))
The time right now is 4:28:02 PM.
\end{lstlisting}

Again, called the same function with the same argument, and got a
different result.

\question{Yeah\dots not being able to fetch the current time, or use
  random numbers sounds insane\dots}

You \emph{can} do those things in Haskell, but you have to make use of
a mathematical structure called a \term{monad}. We'll probably get to
that later.

\paragraph{Installing Haskell} I could spend three hours, and write
out very detailed instructions, with screenshots, for installing
Haskell on every operating, only to have them become obsolete in 6
months. \emph{Or}, I could just direct you to an online resource that
will always be up to date:
\barelink{https://www.haskell.org/downloads}.


\begin{remark}
  It's very important to note that referential transparency simply
  means \ctext{if $a = b$, then $f(a) = f(b)$.} It \xtb{does not} mean
  \ctext{if $f(a) = f(b)$, then $a = b$.} (If that is true, then the
  function is \term{injective}.)\footnote{The word \term{injective} is
    sometimes replaced with \term{one-to-one}.}

  Certainly, not all functions are injective. For instance,
  consider $$f(x) = x^2.$$ We have

  \begin{align*}
    f(5) & = 5^2 \\
         & = 25, \\
  \end{align*}

  and

  \begin{align*}
    f(-5) & = (-5)^2 \\
         & = 25. \\
  \end{align*}

  Therefore, $f(x) = x^2$ is \xtb{not} injective.
\end{remark}

\begin{aside}
  A common problem---``goal'' might be a better word---in mathematics
  is finding the inverse of a function. That is, \ctext{$g$ is the
    \term{left inverse} of $f$ if $g(f(a)) = a$, for all $a$.}

  (You apply it on the left, that's why it's called a left inverse.)

  Let's go back to $$f(x) = x^2.$$ If we let $$g(y) = \sqrt{x},$$ then
  your instinct is to think that $g$ is $f$'s left inverse, because
  $$g(f(x)) = \sqrt{x^2}$$ looks like it \emph{should} be just
  $x$. However,

  \begin{align*}
    g(f(-5)) & = \sqrt{(-5)^2} \\
             & = \sqrt{25} \\
             & = 5
  \end{align*}

  and $5$ is certainly not equal to $-5$. Therefore, $g$ is \xtb{not}
  a left-inverse of $f$.
\end{aside}


\question{You showed above that $f$ is not injective. Is there any
  connection between a function being injective, and the function
  being invertible?}

Yes, there is. In fact, we'll prove later, for a function to be
left-invertible, it needs to be injective.

\question{Is there such a thing as a right inverse?}

Yes. I'll let you try to think of what those might be. There's a
separate condition that needs to hold for a function to be
right-invertible, called \term{surjectivity}. To understand what
surjectivity is, we need to understand a bit more about sets and logic
first. That's part of what this chapter addresses.


\question{The idea of an inverse seems really general. Can we have
  like, an inverse of a number?}

Yes, you can! In fact, there are two general types of inverses:

\begin{enumerate}
\item \term{additive inverses}, like $-x$, and
\item \term{multiplicative inverses}, like $\frac{1}{x}$.
\end{enumerate}

\question{Which category do function inverses fall under?}

Either one, actually. The distinction only applies when you have some
situation in which some objects have one type of inverse, and not the
other, or when they have both types of inverses.

\question{Are the left inverse and the right inverse always equal?}

I'll let you think about it. Hint: there's another condition, which is
true of numbers, that guarantees that the left and right inverses are
the same.

\question{Can you give me an example of some situation where the two
  inverses are not equal?}

I just did, with functions, actually.

\begin{remark}
  I've subtly assumed you know all of the following things:

  \begin{enumerate}
  \item what a ``function'' is,
  \item what an ``argument'' is,
  \item what the notation $f(x)$ means,
  \item what it means for some value $x$ to equal some value $y$
    (i.e. $x = y$).
  \end{enumerate}

  If you don't know what those things are, don't feel bad. I'll
  address these questions with some depth later on in the chapter.
\end{remark}

Here are some more questions that will hopefully be answered in this
chapter

\begin{enumerate}
\item What do you mean when you ``prove'' something in math? Is that
  different than ``proving'' something in real life?
\item How do I construct a mathematical proof, or even know what I'm
  supposed to prove?
\item If I'm proving something mathematically, what logical ``moves''
  am I allowed to make?
\end{enumerate}

That last question might not make a whole lot of sense right now, but
it will later on when you construct some proofs. You'll see that it's
a bit like a game of chess: you have a goal, and you have to figure
out how to get there. There are a limited number of ``moves'' you are
allowed to make to get to your goal. Sometimes, you won't be able to
figure out how to get to your goal. Sometimes, it will turn out that
it's impossible to get to your goal.

\begin{enumerate}[start=4]
\item I learned in school things like, for two numbers,
  $a + b = b + a$. How do we know that's true? Could there be some
  situation where that isn't true?
\item What is a set?
\item If we have two sets, how do we determine which one is
  ``larger''?
\item I've heard of ``cardinal numbers''. What are those? Do they have
  anything to do with the bird?
\item Can we just talk about birds instead of math?
\end{enumerate}

\Cref{s:props-proofs} deals with the first three
questions. \Cref{s:sets}, \cref{s:nats}, and \cref{s:other-numbers}
deal with the next two questions. \Cref{s:cardinality} deals with
questions 6 and 7. The answer to question 8 is ``no''.

\section{Propositions and Boolean logic}
\label{s:props-proofs}
\label{props-proofs}
\label{s:logic}
\label{logic}

Thomas Goller wrote an excellent series of lecture notes for the
discrete math course taught at the University of
Utah. \cite{goller-discrete} In particular, his explanation of ``real
mathematics'' is spot-on. He references some concepts you've likely
never heard of, so I'll paraphrase it here.

Most of ``real mathematics'' is not about computing heinous sums,
doing long division of large numbers, or solving confusing word
problems, all of which you likely did in grade school. It's mostly
about taking mathematical facts, and using them to prove more
interesting mathematical facts.

There are mathematicians who do focus on modeling real-world
situations, or numerically solving difficult problems. However, the
problems they are encountering are far more difficult, and far more
interesting, than the problems you were forced to address in grade
school. Even when mathematicians do focus on real-world phenomena,
they typically invent new methods solving mathematical problems, or
analyze existing methods, rather than just solving the mathematical
problems.

On to ``mathematical facts'': what exactly is a ``mathematical fact''?
Instead of ``mathematical fact'', mathematicians usually use the term
``proposition''. A proposition can either be true or
false.\footnote{This isn't quite true. Sometimes a statement isn't
  true, and it isn't false. We'll address this issue later. For now,
  consider the phrase ``this statement is false.'' This is called the
  liar's paradox.}  They typically take two forms:

\begin{enumerate}
\item \term{Axioms}, also called \term{postulates}, are very simple
  statements that we accept without proof. You have to start
  somewhere.
\item \term{Theorems}, also called \term{lemmas}, are facts that can
  be proven from other axioms or theorems.
\end{enumerate}

Most theorems are proved from other theorems, and not directly from
the axioms. You generally want to minimize the number of axioms you
use, in the interest of making your argument more
believable.\footnote{This is a principle known as ``Occam's Razor''.}
Nothing would ever get done if we had to start from axioms every time.

\subsection{Logical implication}

With that out of the way, let's address our first topic:
\term{implication}. If we want to say that some proposition $A$
implies another proposition $B$, we write $$A \implies B,$$ which
should be read as ``$A$ implies $B$.''\footnote{The ``double arrow''
  notation $A \Rightarrow B$ is far more common. Later, you'll see why
  we use a single arrow.}  You should note that $A \implies B$ is
itself a proposition.

So, what does this mean? Well, $$A \implies B$$ is equivalent to
saying \ctext{if $A$ then $B$.} The expression $B \impliedby A$ means
the same thing as $A \implies B.$ Sometimes, you'll see $$A \iff B,$$
which means \ctext{$A$ implies $B$, \xtb{and} $B$ implies $A$.} This
should be read as ``$A$ if and only if $B$.'' Sometimes, you'll see
the word \ctext{iff,} which isn't a word in English, but it's
shorthand for \ctext{if and only if.}

\begin{remark}
  A common mistake is, if you know $A \implies B,$ and then conclude,
  ``well, if I know $B$ is true, then I can reasonably conclude $A$ is
  true, because $A$ implies $B$.'' That's unfortunately not true.
\end{remark}

\answergraph{images/alligator-attack-1-gimped.png}
\answergraph{images/alligator-attack-2-gimped.png}

\subsection{Logical conjunction and disjunction}

\term{Logical conjunction} is a fancy term for the word
``and''. Sometimes, you'll see the notation $$A \land B,$$ which means
the same thing as \ctext{$A$ logical-and $B$.} In this book, it's easy
enough to just use text, so I'll just write $$A \tand B.$$ However,
when writing by hand, it's quicker to write the $\land$
symbol. There's no real difference between ``logical-and'' and the
normal way you use ``and'' in everyday life.

\emph{However}, there's a huge difference between \term{logical-or},
and the colloquial meaning of ``or,'' which is why we have the prefix
``logical.'' What is ``logical-or,'' then? Consider the phrase
\ctext{$A$ or $B$.} In everyday life, this means that you can either
have $A$, or you can have $B$, but you can't have both, and you can't
have neither. In logic, the phrase \ctext{$A$ or $B$} allows you to
have both. The notion of ``logical-or'' is sometimes called
\term{logical disjunction}. When writing things by hand, the
notation $$A \lor B$$ is popular.

It might be helpful to construct what's called a ``truth table''

\begin{table}[h]
  \centering
  \begin{tabu}[c]{r|cc}
    $\land$ & \True  & \False \\
    \tabucline \\
    \True  & \True  & \False \\
    \False & \False & \False \\
  \end{tabu}
  \caption{Truth table for logical-and}
\end{table}

\begin{table}[h]
  \centering
  \begin{tabu}[c]{r|cc}
    $\lor$ & \True  & \False \\
    \tabucline \\
    \True  & \True  & \True \\
    \False & \True & \False \\
  \end{tabu}
  \caption{Truth table for logical-or}
\end{table}

Here's how we toy around with this in \ghci

In my experience, most people have little to no trouble understanding
these concepts, so I'm going to move on.

\subsection{Properties of logic}

It's time to introduce a number of properties of implication, as well
as properties of logical-and, logical-or, and logical-not. The study
and manipulation of these properties are commonly referred to as
\term{Boolean algebra}, after the English mathematician George Boole.

\begin{axiom}
  Given a proposition $A$, $$A \implies A.$$ This is called the
  \term{reflexive property}.
\end{axiom}

\begin{axiom}
  Given three propositions $A$, $B$, and $C$, if $$A \implies B$$
  and $$B \implies C,$$ then $$A \implies C.$$ This is called the
  \term{transitive property}. In our notation, this is
  $$[(A \implies B)\tand (B \implies C)] \implies (A \implies C).$$
\end{axiom}

Note

\begin{lemma}
  If $$A \implies C,$$ and $$A \iff B,$$ then $$B \implies C.$$ This
  is called the \term{substitution property}.
\end{lemma}
\begin{proof}
  This is actually a special case of the transitive property, and, in
  fact, we only need $A \impliedby B$. In that case, we have
  $B \implies A$, and $A \implies C$, which means that $B \implies C$.
\end{proof}

The white box means ``end of proof''. You'll sometimes see/hear the
phrase ``QED'', which stands for ``quod erat demonstrandum'', Latin
for ``end of proof''.

The proof of this lemma is very similar, and is left as an
exercise. If you don't do the exercises, you should at least convince
yourself that the lemma is true.

\begin{lemma}
  \label{conclusion-substitution}
  If $$A \implies B,$$ and $$B \iff C,$$ then $$A \implies C.$$
\end{lemma}

\begin{lemma}
  Given three propositions $A$, $B$, and $C$,
  $$A \tand (B \tand C) \iff (A \tand B) \tand C.$$ This is called
  \term{the associative property}.
\end{lemma}

\begin{proof}
  Do case analysis on $A$:

  \begin{description}
  \item[Case $A$ is true] In this case, we have, on the left-hand side
    of the $\iff$ symbol, $$\True \tand (B \tand C),$$ which, from our
    truth table above, will be equal to whatever $B \tand C$ is. By
    the substitution property, our proposition is now reduced to
    $$B \tand C \iff (\True \tand B) \tand C.$$ Doing the same thing
    again, $$(\True \tand B) \tand C \iff B \tand C,$$ by our truth
    table. Therefore, we can replace the right-hand side with
    $$B \tand C.$$ Our proposition is now
    $$B \tand C \iff B \tand C,$$ which is true by the reflexive
    property.
  \item[Case $A$ is false] In this case, we have, on the left-hand
    side as $\False \tand (B \tand C)$. If you'll remember from the
    truth table, it doesn't matter what $B \tand C$ is, the entire
    statement is false. Therefore, our proposition is
    $$\False \iff (A \tand B) \tand C.$$ We do the same thing on the
    right-hand side (you have to do it twice) to get
    $$\False \iff \False,$$ which, again, is true by the reflexive property.
  \end{description}
\end{proof}

The rest of these lemmas are more or less ``obvious'', and they can
all be proved by constructing truth tables. (Which is a specialized
version of case analysis). The proofs of them are left as
exercises. Again, if you are too lazy to do the exercises, at least
convince yourself that the lemmas are true.

\begin{lemma}
  \label{and-symmetry}
  Given two propositions $A$ and $B$, $$A \tand B \iff B \tand A.$$
  This is the \term{commutative property}, or the \term{symmetry
    property}.
\end{lemma}

\begin{lemma}
  \label{or-assoc}
  Given two propositions $A$, $B$, and $C$,
  $$A \tor (B \tor C) \iff (A \tor B) \tor C.$$ This is yet another
  iteration of the \term{associative property}, this time for
  logical-or.
\end{lemma}
\begin{lemma}
  \label{or-symmetry}
  Given two propositions $A$ and $B$, $$A \tor B \iff B \tor A.$$
  Guess which property this is.
\end{lemma}

\begin{lemma}
  \label{and-or-dist}
  Given three propositions $A$, $B$, and $C$,
  $$A \tand (B \tor C) \iff (A \tand B) \tor (A \tand C).$$ This is
  the \term{distributive property}.
\end{lemma}

\begin{lemma}
  \label{or-and-dist}
  Given three propositions $A$, $B$, and $C$,
  $$A \tor (B \tand C) \iff (A \tor B) \tand (A \tor C).$$ This is
  another instance of the \term{distributive property}.
\end{lemma}

\subsection{Negation}

If we have some proposition $A$, we might want to say that $A$ is
false. How do we go about doing that? The answer is with logical
negation. We write $$\lnot A$$ to say \ctext{$A$ is false.} The
expression $\lnot A$ should be read as ``not $A$.'' Here's the
definition:

\begin{definition}
  Given a proposition $A$,

  \begin{align*}
    A \text{ is true} & \implies \lnot A \text{ is false, and} \\
    A \text{ is false} & \implies \lnot A \text{ is true.}
  \end{align*}
\end{definition}

Now that you know what negation is, I can explain the
contrapositive. Given a proposition of the form $$A \implies B,$$ its
\term{contrapositive} is $$\lnot B \implies \lnot A.$$ It's not
entirely obvious why a proposition is equivalent to its
contrapositive. I think an example might help.

\begin{example}
  If someone is decapitated, they are dead, at least within a few
  seconds. Therefore, \ctext{Decapitated $\implies$ Dead.} If someone
  is alive (i.e. not dead), you can very reasonably conclude that they
  have not been decapitated. Hence,
  $$\lnot\text{Dead} \implies \lnot\text{Decapitated}.$$
\end{example}

\answergraph{images/alas-poor-yorick.png}

If you are trying to prove something, it's often significantly easier
to prove its contrapositive. This technique is called \term{proof by
  contradiction}. More generally, if you are trying to prove $B$, and
you can prove that $\lnot B$ implies $\lnot A$, but you know $A$ is
true, then you can conclude $B$ is true. $B$ being false would present
a \term{contradiction}. That's why the contrapositive is super
important.

\begin{remark}
  It's worth noting that many logical systems, or just ``logics''
  don't allow for proof by contradiction. We'll explore these
  later. For now, just keep in mind that sometimes, the ideas that
  seem reasonable and ``obvious'' will later turn out to be
  problematic.
\end{remark}

\paragraph{Properties involving negation}

Here's your favorite part: dry listings of ``obvious'' properties,
which you have to prove as exercises!

\begin{lemma}
  \label{double-negation}
  Given a proposition $A$,
  $$A \iff \lnot(\lnot A).$$ This is called \term{double negation}.
\end{lemma}
\begin{lemma}
  \label{de-morgan-1}
  Given two propositions $A$ and $B$,
  $$\lnot(A \tand B) \iff (\lnot A) \tor (\lnot B).$$ This is one of
  \term{De Morgan's laws}.
\end{lemma}
\begin{lemma}
  \label{de-morgan-2}
  Given two propositions $A$ and $B$,
  $$\lnot(A \tor B) \iff (\lnot A) \tand (\lnot B).$$ This is the
  other of \term{De Morgan's laws}.
\end{lemma}

\begin{lemma}
  \label{lem}
  Given a proposition $A$, $$A \tor \lnot A$$ is true. This is called
  \term{the law of excluded middle}.
\end{lemma}

The proof of the law of excluded middle (LEM) is beyond
trivial. However, there's an alternate way of stating LEM, which is
not at all obvious or trivial, but is logically equivalent:

\begin{lemma}
  \label{peirce}
  Given two propositions, $A$ and $B$,
  $$((A \implies B) \implies A) \implies A.$$ This is called
  \term{Peirce's law},\footnote{No, that's not a typo, it's spelled
    ``Peirce''.} and it's equivalent to the law of excluded middle.
\end{lemma}

It's going to take you a little minute to even convince yourself that
that's true. The proof is quite difficult. However, you'll very rarely
ever actually use Peirce's law directly, so it's okay if you can't
figure out how to prove it, and aren't entirely convinced that it's
true. If there comes a day when you do find yourself needing Peirce's
law, you'll likely know logic well enough to prove its equivalence to
the law of excluded middle.

\subsection{Exercises}

\begin{exercise}
  Prove \cref{conclusion-substitution}.
\end{exercise}
\begin{exercise}
  Prove \cref{and-or-dist}.
\end{exercise}
\begin{exercise}
  Prove \cref{or-and-dist}.
\end{exercise}
\begin{exercise}
  Prove \cref{de-morgan-1}.
\end{exercise}
\begin{exercise}
  Prove \cref{de-morgan-2}.
\end{exercise}
\begin{exercise}
  Prove \cref{peirce}.
\end{exercise}

\section{Sets}
\label{s:sets}

In this section, we're going to go over the notion of sets. A set is a
collection of things, called \term{elements}, with no notion of
multiplicity or order. That is, all of the following are the same set:

\begin{tabu} to \textwidth {rX[p]}
  $\mset{1,2,3,4}$ \\
  $\mset{4,2,3,1}$ & The order is different. \\
  $\mset{1,2,4,1,4,3,1,2}$ & The order is different, and some of the elements are repeated. \\
\end{tabu}

To say \ctext{$x$ is an element of $A$,} we write $$x \in A.$$
Sometimes, you'll see a lowercase Greek epsilon ($\epsilon$ or
$\varepsilon$) used instead of $\in$, particularly when writing things
by hand. In the interest of laziness, I (and others) will usually
write write $$x, y \in A$$ in place of something like \ctext{$x \in A$
  and $y \in A$.} To say \ctext{$x$ is \emph{not} an element of $A$,}
you should write $$x \notin A.$$ For example,
$$1 \in \mset{1,2,3,4},$$ but $$5 \notin \mset{1,2,3,4}.$$ In general,
if you have some sort of binary relation, like $\in$, $\implies$, $=$,
$\geq$, etc, crossing out the symbol indicates that it's not true.

\begin{tabu} to \textwidth{X[r,p]l||rX[l,p]}
  $x$ is an element of $A$ & $x \in A$ & $x \notin A$ & $x$ is not an element of $A$ \\
  \tabucline \\
  $P$ implies $Q$ & $P \implies Q$ & $P \notimplies Q$ & $P$ does not imply $Q$ \\
  \tabucline \\
  $a$ is equal to $b$ & $a = b$ & $a \ne b$ & $a$ is not equal to $b$
\end{tabu}

\subsection{Subsets and supersets}

\sirrygraph{images/set-a-subset-b-gimped.png} We want to be able to
say that a certain set is ``contained'' in another set. The notions of
\term{subsets} and \term{supersets} follows.

\begin{definition}
  A set $X$ is an \term{improper subset}---or, just a
  \term{subset}---of $Y$, written
  $$X \subof Y$$ if $$a \in X \implies a \in Y.$$
\end{definition}

\begin{definition}
  A set $X$ is a \term{proper subset} of $Y$, written $$X \psubof Y$$
  if
  $$a \in X \implies a \in Y,$$ and there is some $b \in Y$ such that
  $b \notin X$.
\end{definition}


\question{So, what really is the difference between proper and
  improper subsets?}

An improper subset allows for the two sets to be equal. A proper
subset asserts that, if $A \subof B$, then every element of $A$ is in
$B$, but $B$ has some stuff that $A$ does not.

\question{How do you tell if two sets are equal?}

We'll get to that in a minute, don't worry.

\begin{remark}
  Some people use the symbol $\psubof$ to refer to improper subsets,
  and the something like $\subsetneqq$ to refer to proper subsets.

  It turns out that improper subsets will be a lot more useful than
  proper subsets. Likewise, we'll discover that $\le$ and $\ge$ are
  much more common than $<$ and $>$, respectively.

  That being said it \emph{would} make sense to use the simpler symbol
  $\subset$ for the more common concept.  However, I don't like using
  $\psubof$ to refer to improper subsets, so I won't. It's just
  revolting. It's like when someone scratches their fingernails on a
  chalkboard. It's morally wrong.
\end{remark}

The opposite notion is a ``superset'': \ctext{$Y \supof X$ if and only
  if $X \subof Y$.} The symbol $\psupof$ is defined analogously.

\begin{lemma}
  Every set is an improper subset of itself, but is not a proper
  subset of itself.
\end{lemma}

\question{You said we would define when two sets are equal.}

Alright, here you go. Two sets are equal when they have exactly the
same elements.

\begin{definition}
  Two sets $A$ and $B$ are equal if \ctext{$A \subof B$ and
    $B \subof A$,} that is $$x \in A \iff x \in B.$$
\end{definition}

\begin{aside}
  Sets are kind of fun to play around with, and think of their various
  properties. Hopefully, you can now see why I briefly introduced
  propositional logic. Understanding implication is pretty important
  when discussing sets, particularly with subsets and equality.
\end{aside}

Moving on, there's a set with no elements called the \term{null
  set}. The null set is usually denoted using $\emptyset$,
$\varnothing$, $\phi$, $\Phi$, $\centernot 0$, or some variant
thereof. I'm going to use $\emptyset$.

\begin{lemma}
  The null set is a subset of every set, including itself.
\end{lemma}

\subsection{Set comprehensions}

Before we go much further, we need to introduce some notation, called
``set comprehensions''. For reference, let
$$A = \mset{1,2,3,4,5,6,7,8,9,10}.$$ A set comprehension looks
something like this: $$\scomp{x \in A}{x < 5} = \mset{1,2,3,4},$$ or
maybe $$\scomp{2x \in A}{x \le 5} = \mset{2,4,6,8,10}.$$

\answergraph{images/set-comprehension-gimped.png}

It's alright if you don't quite get the notation yet, you will over
time.

\subsection{Unions, Intersections, and Subtractions}

\sillygraph{images/set-a-b-amb-union-gimped.png} The \term{union} of
two sets is the set of all elements in either set. You could spend ten
minutes trying to understand that wordtastrophe, or you can look at
this definition

\begin{definition}
  The \term{union} of two sets $A$ and $B$ is $$A \cup B = \scomp{x
    \in \Amb}{x \in A \tor x \in B}$$
\end{definition}

\question{What on earth is $\Amb$?}

It's a jaunty $A$. It stands for ``ambient set''. We sort of have to
assume that $A$ and $B$ are both subsets of $\Amb$. I'll explain
\emph{why} later on in the chapter.

\begin{example}
  Let $$\Amb = {1,2,3,4,5,\dots,98,99,100}.$$ Calculate the following:

  \begin{enumerate}
  \item $\mset{1,2,3,4} \union \mset{9,10,11,12}$,
  \item $\mset{12, 13,21,24,30} \union \mset{32,45,62,75,88}$,
  \item $\scomp{x \in \Amb}{x \le 40} \union \scomp{x \in \Amb}{x \ge 60}$,
  \item $\scomp{x \in \Amb}{x \le 60} \union \scomp{x \in \Amb}{x \ge 40}$.
  \end{enumerate}

  \begin{solution}
    \begin{enumerate}
    \item $\mset{1,2,3,4,9,10,11,12}$,
    \item $\mset{12, 13,21,24, 30, 32,45,62,75,88}$,
    \item $\scomp{x \in \Amb}{x \le 40 \tor 60 \le x}$,
    \item $\Amb$.
    \end{enumerate}
  \end{solution}
\end{example}

These properties follow immediately from what we previously proved in
\cref{props-proofs}. You can prove them if you want. I won't.

\begin{lemma}
  Given three sets $A$, $B$, and $C$, $$A \union (B \union C) = (A
  \union B) \union C.$$
\end{lemma}

\begin{lemma}
  Given two sets $A$ and $B$, $$A \union B = B \union A.$$
\end{lemma}

This one is equally obvious, but doesn't necessarily follow from the
properties of logic.

\begin{lemma}
  Given a set $A$, $$A \union \emptyset = A.$$
\end{lemma}

\sirrygraph{images/set-a-b-amb-intersection-gimped.png} The
\term{intersection} of two sets is the set of elements that are in
both sets.

\begin{definition}
  The \term{intersection} of two sets $A$ and $B$ is
  $$A \cap B = \scomp{x \in \Amb}{x \in A \tand x \in B}.$$
\end{definition}

\begin{example}
  Let $$\Amb = {1,2,3,4,5,\dots,98,99,100}.$$ Calculate the following:

  \begin{enumerate}
  \item $\mset{13,34,80,89,91,94} \intersect \mset{13,15,34,69,89,94}$,
  \item $\mset{14,34,38,89,98,100} \intersect \mset{3,9,34,38,98,100}$,
  \item $\scomp{x \in \Amb}{x \le 40} \intersect \scomp{x \in \Amb}{60 \le x}$.
  \item $\scomp{x \in \Amb}{40 \le x} \intersect \scomp{x \in \Amb}{x \le 60}$.
  \end{enumerate}

  \begin{solution}
    \begin{enumerate}
    \item $\mset{13,34,89,94}$
    \item $\mset{34,38,98,100}$,
    \item $\emptyset$
    \item $\scomp{x \in \Amb}{40 \le x \le 60}$.
    \end{enumerate}
  \end{solution}
\end{example}

The following three lemmas are more or less ``obvious'':

\begin{lemma}
  Given a set $A$, $$A \cap \emptyset = \emptyset.$$
\end{lemma}
\begin{lemma}
  Given two sets $A$ and $B$, $$A \cap B = B \cap A.$$
\end{lemma}
\begin{lemma}
  Given three sets $A$, $B$, and $C$, $$A \cap (B \cap C) = (A \cap B)
  \cap C.$$
\end{lemma}

\sillygraph{images/set-a-b-amb-subtraction-gimped.png} \term{Set
  subtraction} is where you take all of the elements in one set that
are not in the other:

\begin{definition}
  The \term{difference} of two sets $A$ and $B$ is
  $$A \setminus B = \scomp{x \in \Amb}{x \in A \tand x \notin B}.$$
\end{definition}



\section{Natural numbers}
\label{s:nats}
\label{s:induction}

\section{Unnatural numbers}
\label{s:unnats}
\label{s:other-numbers}

\section{Functions}
\label{s:functions}

\section{Cardinality}
\label{s:cardinality}


\begin{appendices}
  \chapter{Introduction}

Hi, there! This is a textbook explaining the basic notions in pure
mathematics.  The book currently goes over basic propositional logic,
set theory, and function theory. I eventually want to expand it to go
over arithmetic in a rigorous and fun way.  If you want to jump in, go
to \cref{propositions}.

First of all, what is ``pure mathematics''. A simple definition is
``math that is not applied''. However, that is misleading, because it
might lead you to think that pure mathematics isn't applicable to the
real world. In most cases, pure math is applicable to the real
world.\footnote{That said, in many of those cases, the applications
  have yet to be discovered.} However, when you study pure math, you
usually study it because the math is interesting, not because it's
useful.

\section{Motivations and Ideology}

In American high schools, mathematics is approached as a tool for
computation, and it's almost always studied in the context of
real-world applications. This is even true of most low-level
university courses.

Here's the problem: the real world is actually rather boring. Focusing
on real-world applications usually leads people to think that math is
difficult and boring. Math can be difficult, but if you look at it the
right way, it's definitely not boring.  It's only in the second half
of an undergraduate math track that most students are immersed in pure
mathematics.

Part of this process is usually students going back and re-learning
all of the stuff they learned in high school and early college,
because they didn't learn it with enough mathematical rigor. This
means that high school and the first half of college are usually a
complete waste of math students' time.

\subsection{Target audience}

The target audience is (roughly) people who know basic and
intermediate algebra (i.e. solving static equations) and would like to
gain a deeper understanding of mathematics. This book \xti{might} be
suitable as supplementary material for people learning basic algebra.
However, it's assumed that you know how to solve reasonably simple
static equations.

In particular, I've noticed that there are a large number of
programmers who want to understand advanced topics in computer
science, but don't have a suitable mathematical background to attack
said topics effectively. Most of the early audience of this book is
people in that situation, so it's particularly suited to them.

\subsection{Contributing}

You are welcome---and encouraged---to contribute to this book. The
\LaTeX\ source for the book is on GitHub
\barelink{https://github.com/pharpend/sets-logics}, if you want to
make changes. You can also email the author at
\monospace{<peter.harpending@utah.edu>} if you find an error, or are
confused about something.

This book is free (as in ``unrestricted'') and open-source. It's also
``free'' in the sense that you don't have to pay for it, although
that's far less important.  The official license is the Creative
Commons Attribution 4.0 International License. The full license text
can be found at
\barelink{https://creativecommons.org/licenses/by/4.0/}.

You are encouraged to send your modifications back so others can use
them. You are also encouraged to publish modified versions under the
same license.

\section{Conventions}

Definitions, theorems, axioms, etc are numbered so they can be found
quickly. If I refer to ``definition X.Y.Z'', that means, the Zth
definition, in chapter X, section Y. Definitions, theorems, axioms,
etc. all use the same counter.\footnote{I took this idea from Joseph
  L. Taylor's \xti{Foundations of Analysis}.} For instance,

\begin{definition}
  Here's a definition!
\end{definition}

\begin{example}
  Here's an example.
\end{example}

\begin{axiom}
  Here's an axiom.
\end{axiom}

Exercises don't use the same counter as everything else. They are also
separated by section. For instance,

\begin{exercise}
  Here's an exercise.
\end{exercise}
  \input{chapters/b.tex}
\end{appendices}

\printbibliography
\end{document}
