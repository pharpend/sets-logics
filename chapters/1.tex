\chapter{Propositional Logic}
\label{propositions}

The goal of this chapter is to give you an idea of how to think
mathematically. We introduce the notion of \term{propositions}, and
then of \term{proofs}.

\section{Propositions}

Let's invent a proposition, and call it $p$. $p$ is semantically
equivalent to a ``statement''. A ``proof of $p$'' is a proof that $p$
is true. To say

\begin{zz}
  f \text{ is a proof of } p,
\end{zz}

we write

\begin{zz}
  f : p.
\end{zz}

\begin{remark}
  Outside of certain circles, the $f : p$ notation is very
  non-standard. Later in the book, it will become clear why we use it.

  I often, when writing, use the $:$ symbol to be short for ``such
  that''. Again, you'll see why all of this makes sense later.
\end{remark}

\begin{aside}
  In mathematics, it's very common to see notation you think is
  confusing, only to discover---often several years later---that the
  notation makes perfect sense and is actually brilliant.
\end{aside}

The most basic proofs begin with \term{axioms}---also called
\term{postulates}---and then prove \term{theorems}---also called
\term{lemmas}. However, most proofs incorporate previously proved
lemmas to create new lemmas. If we had to start from the most basic
axioms every time we proved anything, nothing would ever get done.

\begin{definition}
  \label{def:axiom}
  \label{def:postulate}
  An \term{axiom} (or \term{postulate}) is a proposition that we
  accept to be true without proof.
\end{definition}

\begin{remark}
  \label{r:law}
  In most contexts, the term ``law'' is ambiguous. Sometimes, it
  refers to theorems, sometimes it refers to lemmas. I won't call
  anything that I define a ``law''. However, I will call things
  ``laws'' if some vocabulary term is commonly known as ``the law of
  X'' or ``X's law''. (See, for instance, \cref{ax:modus-ponens}).
\end{remark}

\begin{definition}
  \label{def:theorem}
  A \term{theorem} (or \term{lemma}) is a proposition that can be
  proven from other axioms or theorems.
\end{definition}

\subsection{Implication}

The first thing in logic is the notion of ``implication''. We write
$a \implies b$ to say ``$a$ implies $b$''. This means ``if $a$ is
true, then $b$ is also true'', or simply ``if $a$ then $b$''.

\begin{axiom}[Law of Modus Ponens]
  \label{ax:modus-ponens}
  If $a$ implies $b$, and we know $a$ is true, then we can conclude
  $b$.

  In other words, ``$a$ implies $b$. $a$, therefore $b$''.
\end{axiom}

Usually, three dots arranged in a triangle are used to write
``therefore'': $\therefore$.  The statement in \cref{ax:modus-ponens}
can therefore be rewritten as:

\begin{alignmath}{r}
  a \implies b.\; a,\, \therefore b
\end{alignmath}

Sometimes, it's semantically convenient to write $b \impliedby a$,
which just means ``$b$ is implied by $a$''. It's the same as
$a \implies b$. This is also sometimes said ``$b$ if $a$''. This
means, ``assuming $a$, we can prove $b$''.

We'll also often write $a \iff b$ which means that $a$ implies $b$,
\xtb{and} $b$ implies $a$. Usually, you'll hear this pronounced ``$a$
if and only if $b$''. In a minute, we'll get to why people say ``if
and only if'', as opposed to something like ``implies and is implied
by''.

\begin{remark}
  Given two propositions $a$ and $b$, $a \implies b$ is actually a
  proposition. The proposition is, $a$ implies $b$.
\end{remark}

Here are some more or less ``obvious'' properties of logic

\begin{axiom}[Reflexive property]
  \label{ax:props-refl}
  For all propositions $a$, $a \implies a$.
\end{axiom}

\begin{axiom}[Transitive property]
  \label{ax:props-trans}
  For all propositions $a$, $b$, and $c$, if $a \implies b$, and $b
  \implies c$, then $a \implies c$
\end{axiom}

\subsection{Negation}

Quite often, we'll need to declare that something is false. To say
that ``$a$ is false'', you write $\lnot a$, pronounced ``not $a$''.

\begin{axiom}[Double negation]
  The proposition $\lnot(\lnot a)$ is equivalent to $a$.
\end{axiom}

\begin{remark}
  In many (indeed, most) logics, double negation is not accepted as an
  axiom. Many important theorems rely on double negation.

  For instance, the Intermediate Value Theorem in Calculus states
  that, given two points on $a$ and $b$ that are values of a
  ``continuous'' function (a function whose graph does not have any
  gaps or poles), the function touches every single point between $a$
  and $b$. Every known proof of the Intermediate Value Theorem relies
  on double negation. More specifically, it relies on the
  contrapositive, which we'll define in a second.
\end{remark}

\subsection{The contrapositive}

Many logics define two values, $\top$, and $\bot$, called ``top'' and
``bottom'', respectively. $\top$ is always provable, and $\bot$ is
never provable. In those logics, the proposition $\lnot a$ is
equivalent to $a \implies \bot$. If you can prove $a$ implies
something that is never provable, you can conclude that $a$ is false.

This is the common logical technique of \term{reductio ad absurdum},
or \term{proof by contradiction}. ``Reductio ad absurdum'' is Latin
for ``reduction to the absurd''.

\begin{definition}
  Given the proposition $a \implies b$, the proposition
  $\lnot b \implies \lnot a$ is called the \term{contrapositive}.
\end{definition}

\begin{theorem}[Reductio ad absurdum]
  \label{reductio-ad-absurdum}
  Every proposition of the form $a \implies b$ is logically equivalent
  to its contrapositive.
\end{theorem}

The fact that a proposition is logically equivalent to its
contrapositive is not at all obvious. Let's go through some examples.

\begin{example}
  If someone is decapitated, then they are dead (at least within a few
  seconds). Therefore,

  \begin{zz}
    \text{Decapitated} \implies \text{Dead}
  \end{zz}

  \answergraph{images/alas-poor-yorick.png}

  Therefore, if someone is alive, you can conclude that they have not
  been decapitated. Thus,

  \begin{zz}
    \text{not Dead} \implies \text{not Decapitated}
  \end{zz}

  This is the contrapositive.
\end{example}

We will circle back to proving the equivalence between a proposition
and its contrapositive later on. To do this, we need to know what
``equivalence'' is. The next section deals with defining equivalence.

\section{Equivalence}

What exactly does it mean for two propositions to be equivalent? What
does it mean for two numbers to be equivalent? What does it mean for
two shapes to be equivalent?

You are probably tempted to say ``if $p = q$, then $p$ and $q$ are
equivalent''! Problem solved, right?

Well\dots not really. Think about this: the proposition $p \implies q$
is equivalent to $\lnot q \implies \lnot p$. They are clearly not the
same proposition, but they are logically equivalent\dots

How do we universally define equivalence? Think about it for a
second. In fact, I'm going to insert a page break so you can think
about this without cheating and reading the answer. Go for a walk, get
a cup of coffee, whatever it is you do, and think about this\dots

\newpage

Alright, you're back! The answer that mathematicians use is to define
what are called \term{equivalence relations}. What on earth is an
equivalence relation? For that matter, what is a ``relation''?

We're going to explain all of that, as well as define an equivalence
relation for propositions.

Strictly speaking, a ``relation'' is something that takes one or more
objects, and uses them to form a proposition. In this case, we're
going to be studying relations that take two objects, called ``binary
relations''. (The word ``binary'' means ``of or relating to two'').

\begin{definition}
  A \term{binary relation} is something that takes exactly two
  objects, and determines whether or not something is true about them.
\end{definition}

There are, in more advanced areas, relations that require more than
two things. However, for now, when I say ``relation'', I mean
``binary relation''.

\begin{aside}
  There are also \term{unary relations} which take only one object. I
  don't think calling them ``relations'' makes a whole lot of sense,
  because the term ``relation'' implies that you are comparing one
  thing \xti{relative} to other things. If there are no other things
  to compare, the term ``relation'' doesn't make any
  sense. Nonetheless, people use the term all the time.
\end{aside}

You already know quite a few relations.

\begin{enumerate}
\item Given two numbers $a$ and $b$, $=$ is a relation. It takes two
  numbers and decides if they are equal.
\item The symbol $\le$ is also a relation over numbers. It takes two
  numbers, and decides if the first is less than or equal to the
  second.
\item The symbol $\implies$ is also a relation. It takes two
  propositions, and decides if the first implies the second.
\end{enumerate}

What is something we can say about $=$ that we can't say about $\le$?

The answer is that $=$ is an ``equivalence relation''. Meaning that it
satisfies the properties in the next definition

\begin{definition}
  An \term{equivalence relation} is a binary relation $\sim$ satisfying
  these properties:

  \begin{enumerate}
  \item for all $a$, $a \sim a$ (reflexive property),
  \item for all $a$ and $b$, $a \sim b$ implies $b \sim a$ (symmetric
    property),
  \item for all $a$, $b$ and $c$, $a \sim b$ and $b \sim c$ imply
    $a \sim c$ (transitive property).
  \end{enumerate}
\end{definition}

The standard equivalence relation is $=$. It has all of those
properties.

If you'll notice, $\le$ and $\implies$ have all of those properties,
except symmetry.

Is there a relation for propositions that has symmetry? (That is, an
equivalence relation for propositions).

Think about it for a minute. I'll give you another page break\dots

\newpage

Turns out, there is an equivalence relation! The relation is
$\iff$. Think about it:

\begin{description}
\item[Reflexivity] For all propositions $p$, $p \iff
  p$. (\Cref{ax:props-refl})
\item[Symmetry] For all propositions $p$ and $q$, $p \iff q$ implies
  $q \iff p$.
\item[Transitivity] For all propositions $p$, $q$ and $r$, $p \iff q$
  and $q \iff r$ imply $p \iff r$. (\Cref{ax:props-trans})
\end{description}

Can you think of some other equivalence relations? Many of them are
not at all obvious.

\begin{example}
  Analog clocks and watches give the time in 12-hour format. However,
  there are 24 hours in a day. When you say ``14:00 is the same as
  2:00'', you're actually saying ``14 is congruent to 2, modulo
  12''. The way we write this is with

  \begin{zz}
    14 \equiv 2 \mod 12
  \end{zz}

  What exactly is ``modulo''? Technically speaking, $n$ modulo $m$ is
  the remainder of dividing $n$ by $m$. So, $14 / 12 = 1$ with a
  remainder of $2$. We'll get to this more in-depth when we discuss
  arithmetic.

  Here's a fun thing, you can add two times together to get a new
  time. For instance, let's say you go into work at 9:00 in the
  morning, and stay for 8 hours. $9 + 8 = 17$. However,
  $17 \equiv 5 \mod 12$, which means you stay until 5 PM. You've been
  secretly using this equivalence relation your entire life.

  We'll learn later, in the Algebra section, that integers modulo 12
  (i.e. numbers on a clock) form a \term{module} of the integers
  (hence the term ``modulo''). Meaning this:

  \begin{enumerate}
  \item Given two times, $a$ and $b$ in the module, $a + b$ is also in
    the module.
  \item Given an integer\footnote{An integer is a positive, negative
      or zero whole number.} $z$, and a time $t$, $zt$ is also in the
    subgroup.\footnote{The notation $zt$ just means ``$z$ times
      $t$''.}
  \end{enumerate}

  There are a number of other properties that have to hold, but those
  are the properties unique to modules.

  Can you think of some other modules?

  I'll give you a hint: even numbers are also a module.
\end{example}

\subsection{Contrapositive}

Circling back to the contrapositive: I claimed a proposition of the
form $a \implies b$ is equivalent to its contrapositive. Now that
we've defined equivalence, we're much more equipped to prove this.

Two propositions $p$ and $q$ are equivalent if\footnote{Actually, it's
  ``if and only if'', because two propositions $p$ and $q$ being
  ``equivalent'' means $p \iff q$. However, it's conventional to say
  ``if'' when parroting a definition.} $p \iff q$. Therefore,

\begin{lemma}
  \label{thm:contrapositive}
  $(a \implies b) \iff (\lnot b \implies \lnot a)$.
\end{lemma}

\begin{proof}
  The most common way to prove an ``if and only if'' statement is to
  separately prove the $\implies$ branch, and the $\impliedby$ branch.

  \dots
\end{proof}
