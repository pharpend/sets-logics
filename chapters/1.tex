\chapter{Basics}
\label{basics}

The goal of this chapter is to give you an idea of how to think
mathematically. More importantly, you should learn how to express your
ideas mathematically. The language of mathematics is extremely
powerful, and surprisingly simple.

While we are exploring mathematics, I will also display program code
examples from a variety of programming languages, to show how our
mathematical ideas translate over into programming.\footnote{The
  program code examples are licensed under the FreeBSD (2-clause)
  License, a copy of which can be found in \cref{bsd-license}.} In my
experience, learning how to program helps with math, and learning math
helps with programming.

Here are some questions that will hopefully be answered in this
chapter

\begin{enumerate}
\item What do you mean when you ``prove'' something in math? Is that
  different than ``proving'' something in real life?
\item How do I construct a mathematical proof, or even know what I'm
  supposed to prove?
\item If I'm proving something mathematically, what logical ``moves''
  am I allowed to make?
\end{enumerate}

That last question might not make a whole lot of sense right now, but
it will later on when you construct some proofs. You'll see that it's
a bit like a game of chess: you have a goal, and you have to figure
out how to get there. There are a limited number of ``moves'' you are
allowed to make to get to your goal. Sometimes, you won't be able to
figure out how to get to your goal. Sometimes, it will turn out that
it's impossible to get to your goal.

\begin{enumerate}[start=4]
\item I learned in school things like, for two numbers,
  $a + b = b + a$. How do we know that's true? Could there be some
  situation where that isn't true?
\item What is a set?
\item If we have two sets, how do we determine which one is
  ``larger''?
\item I've heard of ``cardinal numbers''. What are those? Do they have
  anything to do with the bird?
\item Can we just talk about birds instead of math?
\end{enumerate}

\Cref{s:props-proofs} deals with the first three
questions. \Cref{s:sets}, \cref{s:nats}, and \cref{s:other-numbers}
deal with the next two questions. \Cref{s:cardinality} deals with
questions 6 and 7. The answer to question 8 is ``no''.

\section{Propositions and Boolean logic}
\label{s:props-proofs}

Thomas Goller wrote an excellent series of lecture notes for the
discrete math course taught at the University of
Utah. \cite{goller-discrete} In particular, his explanation of ``real
mathematics'' is spot-on. He references some concepts you've likely
never heard of, so I'll paraphrase it here.

Most of ``real mathematics'' is not about computing heinous sums,
doing long division of large numbers, or solving confusing word
problems, all of which you likely did in grade school. It's mostly
about taking mathematical facts, and using them to prove more
interesting mathematical facts.

There are mathematicians who do focus on modeling real-world
situations, or numerically solving difficult problems. However, the
problems they are encountering are far more difficult, and far more
interesting, than the problems you were forced to address in grade
school. Even when mathematicians do focus on real-world phenomena,
they typically invent new methods solving mathematical problems, or
analyze existing methods, rather than just solving the mathematical
problems.

On to ``mathematical facts'': what exactly is a ``mathematical fact''?
Instead of ``mathematical fact'', mathematicians usually use the term
``proposition''. A proposition can either be true or
false.\footnote{This isn't quite true. Sometimes a statement isn't
  true, and it isn't false. We'll address this issue later. For now,
  consider the phrase ``this statement is false.'' This is called the
  liar's paradox.}  They typically take two forms:

\begin{enumerate}
\item \term{Axioms}, also called \term{postulates}, are very simple
  statements that we accept without proof. You have to start
  somewhere.
\item \term{Theorems}, also called \term{lemmas}, are facts that can
  be proven from other axioms or theorems.
\end{enumerate}

Most theorems are proved from other theorems, and not directly from
the axioms. You generally want to minimize the number of axioms you
use, in the interest of making your argument more
believable.\footnote{This is a principle known as ``Occam's Razor''.}
Nothing would ever get done if we had to start from axioms every time.

\subsection{Logical implication}

With that out of the way, let's address our first topic:
\term{implication}. If we want to say that some proposition $A$
implies another proposition $B$, we write $$A \implies B,$$ which
should be read as ``$A$ implies $B$.''\footnote{The ``double arrow''
  notation $A \Rightarrow B$ is far more common. Later, you'll see why
  we use a single arrow.}  You should note that $A \implies B$ is
itself a proposition.

So, what does this mean? Well, $$A \implies B$$ is equivalent to
saying \ctext{if $A$ then $B$.} The expression $B \impliedby A$ means
the same thing as $A \implies B.$ Sometimes, you'll see $$A \iff B,$$
which means \ctext{$A$ implies $B$, \xtb{and} $B$ implies $A$.} This
should be read as ``$A$ if and only if $B$.'' Sometimes, you'll see
the word \ctext{iff,} which isn't a word in English, but it's
shorthand for \ctext{if and only if.}

Think about this: $$A \implies B$$ means \ctext{if $A$ then $B$.}
However, this could also be stated \ctext{$B$ if $A$.} Likewise, it
could also be stated \ctext{$A$ only if $B$.} Hence, $A \iff B$ is
usually said as ``$A$ if and only if $B$.''

``If and only if'' can also be thought of as ``is logically equivalent
to''.

\subsection{Logical conjunction and disjunction}

\term{Logical conjunction} is a fancy term for the word
``and''. Sometimes, you'll see the notation $$A \land B,$$ which means
the same thing as \ctext{$A$ logical-and $B$.} In this book, it's easy
enough to just use text, so I'll just write $$A \tand B.$$ However,
when writing by hand, it's quicker to write the $\land$
symbol. There's no real difference between ``logical-and'' and the
normal way you use ``and'' in everyday life.

\emph{However}, there's a huge difference between \term{logical-or},
and the colloquial meaning of ``or,'' which is why we have the prefix
``logical.'' What is ``logical-or,'' then? Consider the phrase
\ctext{$A$ or $B$.} In everyday life, this means that you can either
have $A$, or you can have $B$, but you can't have both, and you can't
have neither. In logic, the phrase \ctext{$A$ or $B$} allows you to
have both. The notion of ``logical-or'' is sometimes called
\term{logical disjunction}. When writing things by hand, the
notation $$A \lor B$$ is popular.

It might be helpful to construct what's called a ``truth table''

\begin{table}[h]
  \centering
  \begin{tabu}[c]{r|cc}
    $\land$ & \True  & \False \\
    \tabucline \\
    \True  & \True  & \False \\
    \False & \False & \False \\
  \end{tabu}
  \caption{Truth table for logical-and}
\end{table}

\begin{table}[h]
  \centering
  \begin{tabu}[c]{r|cc}
    $\lor$ & \True  & \False \\
    \tabucline \\
    \True  & \True  & \True \\
    \False & \True & \False \\
  \end{tabu}
  \caption{Truth table for logical-or}
\end{table}

Most people have little to no trouble understanding these concepts, so
I'm going to move on.


\subsection{Properties of logic}

It's time to introduce a number of properties of implication, as well
as properties of logical-and, logical-or, and logical-not. The study
and manipulation of these properties are commonly referred to as
\term{Boolean algebra}, after the English mathematician George Boole.

\begin{axiom}
  Given a proposition $A$, $$A \implies A.$$ This is called the
  \term{reflexive property}.
\end{axiom}

\begin{axiom}
  Given three propositions $A$, $B$, and $C$, if $$A \implies B$$
  and $$B \implies C,$$ then $$A \implies C.$$ This is called the
  \term{transitive property}. In our notation, this is
  $$(A \implies B)\tand (B \implies C) \implies (A \implies C).$$
\end{axiom}

\begin{lemma}
  If $$A \implies C,$$ and $$A \iff B$$, then $$B \implies C$$. This
  is called the \term{substitution property}.
\end{lemma}
\begin{proof}
  This is actually a special case of the transitive property, and, in
  fact, we only need $A \impliedby B$. In that case, we have
  $B \implies A$, and $A \implies C$, which means that $B \implies C$.
\end{proof}

The white box means ``end of proof''. You'll sometimes see/hear the
phrase ``QED'', which stands for ``quod erat demonstrandum'', Latin
for ``end of proof''.

The proof of this lemma is very similar, and is left as an exercise:

\begin{lemma}
  \label{conclusion-substitution}
  If $$A \implies B,$$ and $$B \iff C,$$ then $$A \implies C.$$
\end{lemma}

The following ``lemmas'' can be proven simply by \term{case
  analysis}. Case analysis is where you separately consider every
possible case of a variable (in this case, a proposition), and then
prove the larger proposition in each individual case. I'll do the
first one. The rest are ``obvious'', to varying degrees. The only ones
that aren't are the distributive laws, the proofs of which are left as
exercises.

\begin{lemma}
  Given three propositions $A$, $B$, and $C$,
  $$A \tand (B \tand C) \iff (A \tand B) \tand C.$$ This is called
  \term{the associative property}.
\end{lemma}

\begin{proof}
  Do case analysis on $A$:

  \begin{description}
  \item[Case $A$ is true] In this case, we have, on the left-hand side
    of the $\iff$ symbol, $$\True \tand (B \tand C),$$ which, from our
    truth table above, will be equal to whatever $B \tand C$ is. By
    the substitution property, our proposition is now reduced to
    $$B \tand C \iff (\True \tand B) \tand C.$$ Doing the same thing
    again, $$\True \tand B \iff B,$$ by our truth table. Therefore, we
    can replace the left-hand side with $$B \tand C.$$ Our proposition
    is now $$B \tand C \iff B \tand C,$$ which is true by the
    reflexive property.
  \item[Case $A$ is false] In this case, we have, on the left-hand
    side as $\False \tand (B \tand C)$. If you'll remember from the
    truth table, it doesn't matter what $B \tand C$ is, the entire
    statement is false. Therefore, our proposition is
    $$\False \iff (A \tand B) \tand C.$$ We do the same thing on the
    right-hand side (you have to do it twice) to get
    $$\False \iff \False,$$ which, again, is true by the reflexive property.
  \end{description}
\end{proof}

\begin{lemma}
  \label{and-symmetry}
  Given two propositions $A$ and $B$, $$A \tand B \iff B \tand A.$$
  This is the \term{commutative property}, or the \term{symmetry
    property}.
\end{lemma}

\begin{lemma}
  \label{or-assoc}
  Given two propositions $A$, $B$, and $C$,
  $$A \tor (B \tor C) \iff (A \tor B) \tor C.$$ This is yet another
  iteration of the \term{associative property}, this time for
  logical-or.
\end{lemma}
\begin{lemma}
  \label{or-symmetry}
  Given two propositions $A$ and $B$, $$A \tor B \iff B \tor A.$$
  Guess which property this is.
\end{lemma}

\begin{lemma}
  \label{and-or-dist}
  Given three propositions $A$, $B$, and $C$,
  $$A \tand (B \tor C) \iff (A \tand B) \tor (A \tand C).$$ This is
  the \term{distributive property}.
\end{lemma}

\begin{lemma}
  \label{or-and-dist}
  Given three propositions $A$, $B$, and $C$,
  $$A \tor (B \tand C) \iff (A \tor B) \tand (A \tor C).$$ This is
  another instance of the \term{distributive property}.
\end{lemma}

\subsection{Negation}

If we have some proposition $A$, we might want to say that $A$ is
false. How do we go about doing that? The answer is with logical
negation. We write $$\lnot A$$ to say \ctext{$A$ is false.} The
expression $\lnot A$ should be read as ``not $A$.''

Now that you know what negation is, I can explain the
contrapositive. Given a proposition of the form $$A \implies B,$$ its
\term{contrapositive} is $$\lnot B \implies \lnot A.$$ It's not
entirely obvious why a proposition is equivalent to its
contrapositive. I think an example might help.

\begin{example}
  If someone is decapitated, they are dead, at least within a few
  seconds. Therefore, \ctext{Decapitated $\implies$ Dead.} If someone
  is alive (i.e. not dead), you can very reasonably conclude that they
  have not been decapitated. Hence,
  $$\lnot\text{Dead} \implies \lnot\text{Decapitated}.$$
\end{example}

\answergraph{images/alas-poor-yorick.png}

If you are trying to prove something, it's often significantly easier
to prove its contrapositive. This technique is called \term{proof by
  contradiction}. More generally, if you are trying to prove $B$, and
you can prove that $\lnot B$ implies $\lnot A$, but you know $A$ is
true, then you can conclude $B$ is true. $B$ being false would present
a \term{contradiction}. That's why the contrapositive is super
important.

\begin{remark}
  It's worth noting that many logical systems, or just ``logics''
  don't allow for proof by contradiction. We'll explore these
  later. For now, just keep in mind that sometimes, the ideas that
  seem reasonable and ``obvious'' will later turn out to be
  problematic.
\end{remark}

\paragraph{Properties involving negation}

Here's your favorite part: dry listings of ``obvious'' properties,
which you have to prove as exercises!

\begin{lemma}
  \label{double-negation}
  Given a proposition $A$,
  $$A \iff \lnot(\lnot A).$$ This is called \term{double negation}.
\end{lemma}
\begin{lemma}
  \label{de-morgan-1}
  Given two propositions $A$ and $B$,
  $$\lnot(A \tand B) \iff (\lnot A) \tor (\lnot B).$$ This is one of
  \term{De Morgan's laws}.
\end{lemma}
\begin{lemma}
  \label{de-morgan-2}
  Given two propositions $A$ and $B$,
  $$\lnot(A \tor B) \iff (\lnot A) \tand (\lnot B).$$ This is the
  other of \term{De Morgan's laws}.
\end{lemma}

\begin{lemma}
  \label{lem}
  Given a proposition $A$, $$A \tor \lnot A$$ is true. This is called
  \term{the law of excluded middle}.
\end{lemma}

The proof of the law of excluded middle (LEM) is beyond
trivial. However, there's an alternate way of stating LEM, which is
not at all obvious or trivial, but is logically equivalent:

\begin{lemma}
  \label{peirce}
  Given two propositions, $A$ and $B$,
  $$((A \implies B) \implies A) \implies A.$$ This is called
  \term{Peirce's law},\footnote{No, that's not a typo, it's spelled
    ``Peirce''.} and it's equivalent to the law of excluded middle.
\end{lemma}

It's going to take you a little minute to even convince yourself that
that's true. The proof is quite difficult. However, you'll very rarely
ever actually use Peirce's law directly, so it's okay if you can't
figure out how to prove it, and aren't entirely convinced that it's
true. If there comes a day when you do find yourself needing Peirce's
law, you'll likely know logic well enough to prove its equivalence to
the law of excluded middle. $\aleph$

\subsection{Exercises}

\begin{exercise}
  Prove \cref{conclusion-substitution}.
\end{exercise}
\begin{exercise}
  Prove \cref{and-or-dist}.
\end{exercise}
\begin{exercise}
  Prove \cref{or-and-dist}.
\end{exercise}
\begin{exercise}
  Prove \cref{de-morgan-1}.
\end{exercise}
\begin{exercise}
  Prove \cref{de-morgan-2}.
\end{exercise}
\begin{exercise}
  Prove \cref{peirce}.
\end{exercise}

\section{Sets}
\label{s:sets}

\section{Natural numbers}
\label{s:nats}
\label{s:induction}

\section{Unnatural numbers}
\label{s:unnats}
\label{s:other-numbers}

\section{Functions}
\label{s:functions}

\section{Cardinality}
\label{s:cardinality}
