\chapter{Basics}
\label{basics}

The goal of this chapter is to give you an idea of how to think
mathematically. More importantly, you should learn how to express your
ideas mathematically. The language of mathematics is extremely
powerful, and surprisingly simple.

While we are exploring mathematics, I will also display program code
examples from a variety of programming languages, to show how our
mathematical ideas translate over into programming.\footnote{The
  program code examples are licensed under the FreeBSD (2-clause)
  License, a copy of which can be found in \cref{bsd-license}.} In my
experience, learning how to program helps with math, and learning math
helps with programming.

Here are some questions that will hopefully be answered in this
chapter

\begin{enumerate}
\item What do you mean when you ``prove'' something in math? Is that
  different than ``proving'' something in real life?
\item How do I construct a mathematical proof, or even know what I'm
  supposed to prove?
\item If I'm proving something mathematically, what logical ``moves''
  am I allowed to make?
\end{enumerate}

That last question might not make a whole lot of sense right now, but
it will later on when you construct some proofs. You'll see that it's
a bit like a game of chess: you have a goal, and you have to figure
out how to get there. There are a limited number of ``moves'' you are
allowed to make to get to your goal. Sometimes, you won't be able to
figure out how to get to your goal. Sometimes, it will turn out that
it's impossible to get to your goal.

\begin{enumerate}[start=4]
\item I learned in school things like, for two numbers,
  $a + b = b + a$. How do we know that's true? Could there be some
  situation where that isn't true?
\item What is a set?
\item If we have two sets, how do we determine which one is
  ``larger''?
\item I've heard of ``cardinal numbers''. What are those? Do they have
  anything to do with the bird?
\item Can we just talk about birds instead of math?
\end{enumerate}

\Cref{s:props-proofs} and \cref{s:negation} deal with the first three
questions. \Cref{s:sets}, \cref{s:nats}, and \cref{s:other-numbers}
deal with the next two questions. \Cref{s:cardinality} deals with
questions 6 and 7. The answer to question 8 is ``no''.

\section{Propositions \& Proofs}
\label{s:props-proofs}

Thomas Goller wrote an excellent series of lecture notes for the
discrete math course taught at the University of
Utah. \cite{goller-discrete} In particular, his explanation of ``real
mathematics'' is spot-on. He references some concepts you've likely
never heard of, so I'll paraphrase it here.

Most of ``real mathematics'' is not about computing heinous sums,
doing long division of large numbers, or solving confusing word
problems, all of which you likely did in grade school. It's mostly
about taking mathematical facts, and using them to prove more
interesting mathematical facts.

There are mathematicians who do focus on modeling real-world
situations, or numerically solving difficult problems. However, the
problems they are encountering are far more difficult, and far more
interesting, than the problems you were forced to address in grade
school. Even when mathematicians do focus on real-world phenomena,
they typically invent new methods solving mathematical problems, or
analyze existing methods, rather than just solving the mathematical
problems.

On to ``mathematical facts'': what exactly is a ``mathematical fact''?
Instead of ``mathematical fact'', mathematicians usually use the term
``proposition''. A proposition can either be true or
false.\footnote{This isn't quite true. Sometimes a statement isn't
  true, and it isn't false. We'll address this issue later. For now,
  consider the phrase ``this statement is false.'' This is called the
  liar's paradox.}  They typically take two forms:

\begin{enumerate}
\item \term{Axioms}, also called \term{postulates}, are very simple
  statements that we accept without proof. You have to start
  somewhere.
\item \term{Theorems}, also called \term{lemmas}, are facts that can
  be proven from other axioms or theorems.
\end{enumerate}

Most theorems are proved from other theorems, and not directly from
the axioms. You generally want to minimize the number of axioms you
use, in the interest of making your argument more
believable.\footnote{This is a principle known as ``Occam's Razor''.}
Nothing would ever get done if we had to start from axioms every time.

\subsection{Logical implication}

With that out of the way, let's address our first topic:
\term{implication}. If we want to say that some proposition $A$
implies another proposition $B$, we write $$A \implies B,$$ which
should be read as ``$A$ implies $B$.''\footnote{The ``double arrow''
  notation $A \Rightarrow B$ is far more common. Later, you'll see why
  we use a single arrow.}  You should note that $A \implies B$ is
itself a proposition.

So, what does this mean? Well, $$A \implies B$$ is equivalent to
saying \ctext{if $A$ then $B$.} The expression $B \impliedby A$ means
the same thing as $A \implies B.$ Sometimes, you'll see $$A \iff B,$$
which means \ctext{$A$ implies $B$, \xtb{and} $B$ implies $A$.} This
should be read as ``$A$ if and only if $B$.'' Sometimes, you'll see
the word \ctext{iff,} which isn't a word in English, but it's
shorthand for \ctext{if and only if.}

Think about this: $$A \implies B$$ means \ctext{if $A$ then $B$.}
However, this could also be stated \ctext{$B$ if $A$.} Likewise, it
could also be stated \ctext{$A$ only if $B$.} Hence, $A \iff B$ is
usually said as ``$A$ if and only if $B$.''

\subsection{Logical conjunction and disjunction}

\term{Logical conjunction} is a fancy term for the word
``and''. Sometimes, you'll see the notation $$A \land B,$$ which means
the same thing as \ctext{$A$ logical-and $B$.} In this book, it's easy
enough to just use text, so I'll just write $$A \tand B.$$ There's no
real difference between ``logical-and'' and the normal way you use
``and'' in everyday life.

\emph{However}, there's a huge difference between \term{logical-or},
and the colloquial meaning of ``or,'' which is why we have the prefix
``logical.'' What is ``logical-or,'' then? Consider the phrase
\ctext{$A$ or $B$.} In everyday life, this means that you can either
have $A$, or you can have $B$, but you can't have both, and you can't
have neither. In logic, the phrase \ctext{$A$ or $B$} allows you to
have both.

\section{Negation}
\label{s:negation}

\section{Sets}
\label{s:sets}

\section{Natural numbers}
\label{s:nats}
\label{s:induction}

\section{Unnatural numbers}
\label{s:unnats}
\label{s:other-numbers}

\section{Functions}
\label{s:functions}

\section{Cardinality}
\label{s:cardinality}