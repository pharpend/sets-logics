\chapter{Basics}
\label{basics}

The goal of this chapter is to give you an idea of how to think
mathematically. More importantly, you should learn how to express your
ideas mathematically. The language of mathematics is extremely
powerful, and surprisingly simple.

While we are exploring mathematics, I will also display program code
examples from a variety of programming languages, to show how our
mathematical ideas translate over into programming.\footnote{The
  program code examples are licensed under the FreeBSD (2-clause)
  License, a copy of which can be found in \cref{bsd-license}.} In my
experience, learning how to program helps with math, and learning math
helps with programming.

Here are some questions that will hopefully be answered in this
chapter

\begin{enumerate}
\item What do you mean when you ``prove'' something in math? Is that
  different than ``proving'' something in real life?
\item How do I construct a mathematical proof, or even know what I'm
  supposed to prove?
\item If I'm proving something mathematically, what logical ``moves''
  am I allowed to make?
\end{enumerate}

That last question might not make a whole lot of sense right now, but
it will later on when you construct some proofs. You'll see that it's
a bit like a game of chess: you have a goal, and you have to figure
out how to get there. There are a limited number of ``moves'' you are
allowed to make to get to your goal. Sometimes, you won't be able to
figure out how to get to your goal. Sometimes, it will turn out that
it's impossible to get to your goal.

\begin{enumerate}[start=4]
\item I learned in school things like, for two numbers,
  $a + b = b + a$. How do we know that's true? Could there be some
  situation where that isn't true?
\item What is a set?
\item If we have two sets, how do we determine which one is
  ``larger''?
\item I've heard of ``cardinal numbers''. What are those? Do they have
  anything to do with the bird?
\item Can we just talk about birds instead of math?
\end{enumerate}

\Cref{s:props-proofs} and \cref{s:negation} deal with the first three
questions. \Cref{s:sets}, \cref{s:nats}, and \cref{s:other-numbers}
deal with the next two questions. \Cref{s:cardinality} deals with
questions 6 and 7. The answer to question 8 is ``no''.

\section{Propositions \& Proofs}
\label{s:props-proofs}

\section{Negation}
\label{s:negation}

\section{Sets}
\label{s:sets}

\section{Natural numbers}
\label{s:nats}
\label{s:induction}

\section{Unnatural numbers}
\label{s:unnats}
\label{s:other-numbers}

\section{Functions}
\label{s:functions}

\section{Cardinality}
\label{s:cardinality}