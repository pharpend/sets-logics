\chapter{Sets}

Sets are one of the most basic mathematical tools. The purpose of this
chapter is to define sets, and do basic reasoning about them. We're
also going to give a loose definition of common sets we deal
with. Spoiler: it will later turn out that much of what we do in this
chapter is actually wrong, and can lead to some
contradictions.\footnote{If you're really impatient, and can't wait
  for another chapter, you can read the \link{Wikipedia page on
    Russell's
    paradox}{https://en.wikipedia.org/wiki/Russell\%27s_paradox}}

Sets are collections of objects. The easiest way to denote sets is to
list their elements between curly braces: $\braces{}$.

\begin{displaymath}
  \mset{1,2,3,4}
\end{displaymath}

There is no notion of duplication or order, so all of the following
sets are the same

\begin{tabu}{rp{5.5cm}}
  $\mset{1,2,3,4}$         & \\
  $\mset{1,1,2,2,3,3,4,4}$ & (Each element is duplicated.) \\
  $\mset{2,1,4,3,4,4,1}$   & (The order is different, and some of the elements are duplicated.)
\end{tabu}

There's a set with no elements, called the ``null set'', or the
``empty set'', or a variety of other names. It's usually denoted
$\emptyset$, $\phi$, or some variant thereof. I'm going to use $\nil$.

First of all, if we have a set $A$ and an object $x$, and $x$ is an
element of $A$, we'll write $x \in A$. Note: I've have an old textbook
that uses the Greek letter epsilon ($\epsilon$ or $\varepsilon$) in
place of $\in$. The $\in$ symbol kind of looks like an `E', which you
can mentally associate with ``element''. The symbol $\in$ can be read
as ``is an element of''. If some potential element $y$ is \xtb{not} in
$A$, then we'll write $y \notin A$.\footnote{As a general rule, if you
  have some symbol that indicates that something is true, crossing out
  the symbol indicates that thing is false.}

Continuing with the above example, we have the set
$\mset{1,2,3,4}$. We have

\begin{displaymath}
  1 \in \mset{1,2,3,4}
\end{displaymath}

but

\begin{displaymath}
  5 \notin \mset{1,2,3,4}
\end{displaymath}

We can have sets that have infinitely many elements. For instance, the
``natural numbers'', usually denoted with $\N$, are
infinite.\footnote{Conventions differ on whether or not $\N$ includes
  $0$. It doesn't matter all that much, especially as far as this book
  is concerned.} These are positive whole numbers (i.e. not fractions
or decimals).

\begin{displaymath}
  \N = \mset{1,2,3,4,\dots}
\end{displaymath}

The ``integers'' are any whole numbers, which include $0$ and negative
numbers. The integers are usually denoted with $\Z$. The $\Z$ stands
for ``Zahlen'', which is German for ``numbers''.\footnote{I heard this
  from a professor once. If it's wrong, blame him.}

\begin{displaymath}
  \Z = \mset{\dots, -3, -2, -1, 0, 1, 2, 3, \dots}
\end{displaymath}

\section{Set comprehensions}

Before we go much further, I want to introduce the notion of a ``set
comprehension'', or ``set builder notation''. The basic format is this:

\begin{displaymath}
  \scomp{\text{\xtb{What each element looks like}}}{\text{\xtb{Conditions}}}
\end{displaymath}

The ``conditions'' are just things that have to hold true about the
element.

Here are some examples:

\begin{displaymath}
  \scomp{x \in \N}{x < 5} = \mset{1,2,3,4}
\end{displaymath}
\begin{displaymath}
  \scomp{2x \in \N}{x < 5} = \mset{2,4,6,8}
\end{displaymath}
\begin{displaymath}
  \scomp{x \in \Z}{x < 5} = \mset{\dots,-3,-2,-1,0,1,2,3,4}
\end{displaymath}

Each set comprehension should be read in two parts: the part before
the colon, and the part after the colon.

\begin{itemize}
\item $x \in \N$ means that we are choosing\footnote{Yes, we'll get to
    the axiom of choice later. Calm down.} elements of $\N$, and we're
  calling the element $x$.
\item $x < 5$ means that we are choosing all elements of $\N$ which
  are less than $5$. In the part before the colon, we are choosing one
  such element as an example, and calling it $x$.
\end{itemize}

\subsection{Trying these on your computer}

Many programming languages have ``list comprehensions'', which are
conceptually similar to set comprehensions. For instance, in Haskell,
we can construct the equivalent of our sets from above like this:

\begin{lstlisting}[language=Haskell,caption={The sets from above, in Haskell}]
ghci> [ x | x <- [1..4] ]
[1,2,3,4]
ghci> [ 2*x | x <- [1..4] ]
[2,4,6,8]
\end{lstlisting}

In Python, the syntax is a bit different, but:

\begin{lstlisting}[language=Python,caption={The same thing in Python}]
>>> [ x for x in range(1,5) ]
[1, 2, 3, 4]
>>> [ 2*x for x in range(1,5) ]
[2, 4, 6, 8]
\end{lstlisting}

You probably already have Python installed on your system. Open a
terminal, and run the command \monospace{python} (hit Enter after
typing ``python''). If not, try following the instructions here:
\barelink{https://wiki.python.org/moin/BeginnersGuide/Download}.

You probably do \xtb{not} have Haskell installed on your system,
unless you installed it on purpose. You can find documentation on how
to install Haskell here: \barelink{http://haskellstack.org/}.

\subsubsection{Lists v. Sets}

Lists are different from sets, because, in a list, order and
duplication matter. You can test this out in Haskell like this:

\begin{lstlisting}[language=Haskell,caption={List equality in Haskell}]
ghci> [1,2,3,4] == [1,2,3,4]
True
ghci> [1,2,3,4] == [4,3,2,1]
False
ghci> [1,2,3,4] == [1,2,3,1,3,3,1,4,2]
False
\end{lstlisting}

Haskell actually has a type for sets, called \monospace{Set}. To turn
a list into a set, you use the function \monospace{fromList}

\begin{lstlisting}[language=Haskell,caption={Set equality in Haskell}]
ghci> import Data.Set
ghci> fromList [1,2,3,4] == fromList [1,2,3,4]
True
ghci> fromList [1,2,3,4] == fromList [4,3,2,1]
True
ghci> fromList [1,2,3,4] == fromList [1,4,3,1,2]
True
\end{lstlisting}

\section{Rational and Real Numbers}

The reason I introduced the set comprehension notation is because (1)
it's useful and important, and (2) it's convenient to define the
rational numbers (numbers that can be expressed as a ratio of two
integers) like this:

\begin{displaymath}
  \Q = \scomp{\frac{x}{y} \in \R}{x, y \in \Z; y \ne 0}
\end{displaymath}

Remember, kids: \xtb{you can't divide by zero}, hence the $y \ne 0$
part. The notation $x, y \in \Z$ is just shorthand for $x \in \Z$ and
$y \in \Z$. Mathematicians are incredibly lazy, so you'll often see
confusing notational shorthand used as a replacement for actually
explaining an idea. The symbol $\R$ refers to the ``real
numbers''. The symbol $\Q$ stands for ``quotient''.

Rational numbers are numbers that can be expressed as a ratio of two
integers. Real numbers turn out to be really hard to define. You can
for now think of a real number as any number that can be written down,
supposing one had infinite paper, ink, and time. For instance, $\pi$
cannot be expressed as the ratio of two integers, but is a real
number.

\subsection{Numerical representations of Real numbers}

As it turns out, this definition of real numbers doesn't work! The
reason is, there are cases where you write down two different numbers,
and they refer to the same number.

(I stole this example from a professor of mine, Peter Alfeld, who
presumably stole it from someone else.\cite{pa-unique})

\begin{example}[Issues with numerical representations of real numbers]
  Consider the number $x = 0.9999999\dots$. We can multiply by
  $10$. To do so, we just move everything one digit to the left.

  \begin{rclmath}
      x & = & 0.99999\dots \\
    10x & = & 9.99999\dots \\
  \end{rclmath}

  Well, we can take $9x = 10x - x$, and get this

  \begin{rclmath}
    10x & = & 9.99999\dots \\
      x & = & 0.99999\dots \\
     9x & = & 9 \\
  \end{rclmath}

  We get this because every single digit to the right of the decimal
  point cancels. We can take $x = \frac{9x}{9}$ and get

  \begin{rclmath}
     9x & = & 9 \\
      x & = & 1 \\
  \end{rclmath}

  This is a silly example, and doesn't constitute a ``proof''. However,
  it should give you a glimpse at why numerical representations of real
  numbers are problematic.
\end{example}

Here's another example, which I stole from LeRoy Eide (he showed me
this in person).

\begin{example}[Issues with numerical representations of integers]
  Take a number that begins with an endless string of $9$s to the
  left.

  \begin{displaymath}
    x = \dots99994
  \end{displaymath}

  Let's take $x + 6$.

  \begin{displaymath}
    \begin{tabu}{rr}
        & \dots99994 \\
      + & 6 \\
      \tabucline \\
      = & 0 \\
    \end{tabu}
  \end{displaymath}

  Try it! You add the $6$ to the $4$, and get $10$. Fair enough, put
  $0$ and carry the $1$

  \tabulinesep=0ex
  \begin{displaymath}
    \begin{tabu}{rr}
        & \text{1\hspace{2.1mm}}  \\
        & \dots99994 \\
      + & 6 \\
      \tabucline \\
      = & 00 \\
    \end{tabu}
  \end{displaymath}

  Let's try it again

  \begin{displaymath}
    \begin{tabu}{rr}
        & \text{1\hspace{0.9mm}1\hspace{2.1mm}}  \\
        & \dots99994 \\
      + & 6 \\
      \tabucline \\
      = & 000 \\
    \end{tabu}
  \end{displaymath}

  Continue this process...

  \begin{displaymath}
    \begin{tabu}{rr}
        & \text{\dots 1\hspace{0.9mm}1\hspace{0.9mm}1\hspace{0.9mm}1\hspace{2.1mm}}  \\
        & \dots99994 \\
      + & 6 \\
      \tabucline \\
      = & \dots 00000 \\
    \end{tabu}
  \end{displaymath}

  Therefore, we conclude that $\dots999994 + 6 = 0$, and therefore
  that $\dots99994 = -6$.
\end{example}

Note, we do use infinite carry all the time. For instance, it's why

\begin{displaymath}
  \frac{2}{3} + \frac{1}{3} = 0.666\dots + 0.333\dots = 0.999\dots = 1
\end{displaymath}

Infinite carry actually follows nicely from something called
``mathematical induction''. I'll explain induction in a later chapter.

\begin{example}(More insanity with numbers)
  We can take this a step further, and combine the two cases.

  Let's look at $x = \dots99994$, and examine what happens when we
  take $\frac{x}{100}$.

  \begin{displaymath}
    \frac{x}{100} = \dots999.94
  \end{displaymath}

  Let's look at $\frac{x}{10}$

  \begin{displaymath}
    \frac{x}{10} = \dots999.4
  \end{displaymath}

  Well, remember $\frac{x}{10} = \frac{10x}{100}$. Let's let
  $y = \frac{x}{100}$. Therefore, $10y = \frac{x}{10}$. Let's do what
  we did before, and take $10y - y$

  \begin{displaymath}
    \begin{tabu}{rr}
        & \dots999.4\text{\hspace{2mm}} \\
      - & \dots999.94 \\
      \tabucline \\
      = & -0.54 \\
      = & -\frac{54}{100} \\
    \end{tabu}
  \end{displaymath}

  Now we have that $9y = -\frac{54}{100}$. Therefore, $y =
  -\frac{6}{100} = -\frac{3}{50}$. However, we had that $y =
  \frac{x}{10}$, and therefore $10y = x$. Thus, we have

  \begin{displaymath}
    x = -\frac{3}{5} = -0.6
  \end{displaymath}

  Remember, before, we proved $x = -6$.
\end{example}

The problem here is that \xtb{decimal (base $10$) representations of
  real numbers are not unique}. That sounds like a contradiction, but
it really isn't. It turns out, and we might prove this later, that you
can tweak the rules of writing numbers down to make decimal
representations unique. For decimal numbers, we assert that the number
can't begin or end with an infinite string of $9$s.

\section{Unions and Intersections}
\section{Differences}
\section{De Morgan's Laws}
