\chapter{Introduction}

Hi, there! This is a textbook explaining the basic notions in pure
mathematics.  The book currently goes over basic propositional logic,
set theory, and function theory. I eventually want to expand it to go
over arithmetic in a rigorous and fun way.  If you want to jump in, go
to \cref{propositions}.

First of all, what is ``pure mathematics''. A simple definition is
``math that is not applied''. However, that is misleading, because it
might lead you to think that pure mathematics isn't applicable to the
real world. In most cases, pure math is applicable to the real
world.\footnote{That said, in many of those cases, the applications
  have yet to be discovered.} However, when you study pure math, you
usually study it because the math is interesting, not because it's
useful.

\section{Motivations and Ideology}

In American high schools, mathematics is approached as a tool for
computation, and it's almost always studied in the context of
real-world applications. This is even true of most low-level
university courses.

Here's the problem: the real world is actually rather boring. Focusing
on real-world applications usually leads people to think that math is
difficult and boring. Math can be difficult, but if you look at it the
right way, it's definitely not boring.  It's only in the second half
of an undergraduate math track that most students are immersed in pure
mathematics.

Part of this process is usually students going back and re-learning
all of the stuff they learned in high school and early college,
because they didn't learn it with enough mathematical rigor. This
means that high school and the first half of college are usually a
complete waste of math students' time.

\subsection{Target audience}

The target audience is (roughly) people who know basic and
intermediate algebra (i.e. solving static equations) and would like to
gain a deeper understanding of mathematics. This book \xti{might} be
suitable as supplementary material for people learning basic algebra.
However, it's assumed that you know how to solve reasonably simple
static equations.

In particular, I've noticed that there are a large number of
programmers who want to understand advanced topics in computer
science, but don't have a suitable mathematical background to attack
said topics effectively. Most of the early audience of this book is
people in that situation, so it's particularly suited to them.

\subsection{Contributing}

You are welcome---and encouraged---to contribute to this book. The
\LaTeX\ source for the book is on GitHub
\barelink{https://github.com/pharpend/sets-logics}, if you want to
make changes. You can also email the author at
\monospace{<peter.harpending@utah.edu>} if you find an error, or are
confused about something.

This book is free (as in ``unrestricted'') and open-source. It's also
``free'' in the sense that you don't have to pay for it, although
that's far less important.  The official license is the Creative
Commons Attribution 4.0 International License. The full license text
can be found at
\barelink{https://creativecommons.org/licenses/by/4.0/}.

You are encouraged to send your modifications back so others can use
them. You are also encouraged to publish modified versions under the
same license.

\section{Conventions}

Definitions, theorems, axioms, etc are numbered so they can be found
quickly. If I refer to ``definition X.Y.Z'', that means, the Zth
definition, in chapter X, section Y. Definitions, theorems, axioms,
etc. all use the same counter.\footnote{I took this idea from Joseph
  L. Taylor's \xti{Foundations of Analysis}.} For instance,

\begin{definition}
  Here's a definition!
\end{definition}

\begin{example}
  Here's an example.
\end{example}

\begin{axiom}
  Here's an axiom.
\end{axiom}

Exercises don't use the same counter as everything else. They are also
separated by section. For instance,

\begin{exercise}
  Here's an exercise.
\end{exercise}